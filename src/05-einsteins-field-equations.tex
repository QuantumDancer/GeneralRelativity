\chapter{Einstein's Field Equations}
We will derive Einsteins Equations by physical considerations. The poisson
equation reads as:
\begin{equation}
\Delta\Phi=4\pi\rho
\end{equation}
\begin{equation}
S\textsubscript{EH}=\int\dif x^4 \sqrt{-g}(R-2\Lambda)
\end{equation}
In SR, the energy impuls tensor $\tensor{T}{_\mu_\nu}$ is conserved i.e.
$\tensor{T}{_\mu_\nu^{,\nu}}=0$.As a natural extension we demand that the energy
impuls tensor of general relativity is \emph{covariantly} conserved 
\begin{equation}
\tensor{\nabla}{^\nu}\tensor{T}{_\mu_\nu}=\tensor{T}{_\mu_\nu^{;\nu}}=0
\end{equation}
We now ask for the most general tensor satisfying this equation. 
\begin{theorem}[Lovelock]
For a fourdimensional space the most general divergence free tensor
$\tensor{A}{_\mu_\nu}$ is given by
\begin{equation}
\tensor{A}{_\mu_\nu}= c_1\tensor{G}{_\mu_\nu}+c_2\tensor{g}{_\mu_\nu}\, .
\end{equation}
Where $\tensor{G}{_\mu_\nu}$ is the \emph{Einstein tensor}
$\tensor{G}{_\mu_\nu}:=\tensor{R}{_\mu_\nu}-\frac{1}{2}\tensor{g}{_\mu_\nu}R$.
\end{theorem}
The Theorem imediatly implies that 
\begin{equation}
\tensor{R}{_\mu_\nu}-\frac{1}{2}R\tensor{g}{_\mu_\nu}-\Lambda\tensor{g}{_\mu_\nu}=\kappa\tensor{T}{_\mu_\nu}
\end{equation}
for some constants $\kappa$, $\Lambda$.
Of course we identify $\kappa=\frac{8\pi}{c^2}$ and $\Lambda$ is the
cosmological constant. We can rewrite equation (??) as
\begin{equation}
\tensor{R}{_\mu_\nu}-\frac{1}{2}R\tensor{g}{_\mu_\nu}
=\kappa\left(\tensor{T}{_\mu_\nu}-\frac{\Lambda}{\kappa}\tensor{g}{_\mu_\nu}\right)
\end{equation}
so that the left hand side represents the geometrical part and the right hand
side the matter content. 
Wheeler: ``Geometry tells matter how to move, matter tells geometry how to
curve.''
\begin{sidenote}
In the time of the inflation the cosmological constant must have been large.
Since it is small today it has do decay with time
\end{sidenote}
We can identfy $\Lambda$ with an vacuum energy so that. Can we get the Einstein
equations from a variation principle?
\begin{equation}
S\textsubscript{g}=\int\dif x^4 \sqrt{-g}\tilde{\mathcal{L}}
\end{equation}
$\tilde{\mathcal{L}}$ must transform a (scalar) density, therefore we define a
scalar $\sqrt{-g}\mathcal{L}=\mathcal{L}$
\begin{equation}
S\textsubscript{g}=\int\dif x^4 \sqrt{-g}\mathcal{L}
\end{equation}
One can think of various contributions to $\mathcal{L}$
\begin{equation*}
R,\, \square
R,\,\tensor{\nabla}{^\mu}\tensor{\nabla}{^\mu}\tensor{R}{_\mu_\nu},\,
\tensor{R}{_\mu_\nu}\tensor{R}{^\mu^\nu}
,\,\tensor{R}{_\mu_\nu_\sigma_\rho}\tensor{R}{^\mu^\nu^\sigma^\rho}\dots
\end{equation*}
Which have to be contracted so that the resulting quantity becomes a scalar.
We have no contributions of the metric because
$\tensor{g}{_\mu_\nu_{;\sigma}}=0$. From Yang-Mills theory one would expect a
structure 
\begin{equation}
\mathcal{L}\sim\tensor{F}{_\mu_\nu}\tensor{F}{^\mu^\nu}
\end{equation}
but $\Gamma$ is not the fundamental field but $g$ is. If we demand that we only
have up to second derivatives of $g$ the only allowed term in the Lagrangian is
$R$.
This constraint leads to the \emph{Einstein-Hilbert-action}
\begin{equation}
S\textsubscript{EH}=\frac{1}{2\kappa}\int\dif x^4 \sqrt{-g}(R-2\Lambda)
\end{equation}
\begin{sidenote}[On higher derivatives]
If we include higher order derivatives of $g$ in the right way we can make the
resulting theory renormalisable. However we violate unitarity and introduce so
called ghost fields which are associated with the additional degrees of freedom
we get.
\end{sidenote}
 