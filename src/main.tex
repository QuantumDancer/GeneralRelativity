% !TEX program = pdflatex
% !TEX encoding = UTF-8

\documentclass[a4paper,oneside,12pt]{scrreprt}

%%% PACKAGES + MODIFICATIONS%%%
 
%language and encoding
\usepackage[utf8]{inputenc}
\usepackage[T1]{fontenc}
%\usepackage{lmodern}
\usepackage[english,ngerman]{babel}

%graphics
\usepackage{graphicx}
\usepackage{xcolor}

%math
\usepackage{amsmath}
\usepackage{amssymb}
\usepackage{amsfonts}
\usepackage{amsthm}
\usepackage{thmtools}
\usepackage{mathtools}
\usepackage{tensor}
\usepackage{commath}
\usepackage{dsfont}		% double stroke characters
\usepackage{braket}
\usepackage{mdframed}	% framed environments that can split at page bound­aries

%science
\usepackage{units}
\usepackage{bpchem}	% type­set chem­i­cal names, for­mu­lae, etc

%layout + style
\definecolor{section_color}{rgb}{0.35,0.0,0}								% colors
\definecolor{MyGray}{rgb}{0.96,0.97,0.98}

\usepackage[bottom]{footmisc}
\usepackage[automark,headsepline]{scrlayer-scrpage} 						% headings
\renewcommand*{\headfont}{\normalfont}										% nicht kursive Kopfzeile
\usepackage{setspace}														% set space be­tween lines
\usepackage[font=small,labelfont=bf,labelsep=endash,format=plain]{caption}	% change style of captions
\usepackage[section]{placeins}												% de­fines a \FloatBar­rier com­mand, be­yond which floats may not pass.
\usepackage[protrusion=true,expansion=true]{microtype}						% sublim­i­nal re­fine­ments to­wards ty­po­graph­i­cal per­fec­tion
\addtokomafont{caption}{\small\linespread{1}\selectfont}					% Ändert Schriftgröße und Zeilenabstand bei captions
\usepackage{enumerate} % better way to config enumerates
\usepackage{pdflscape} % landscape mode
\usepackage{afterpage}
% styles for math environments
\declaretheoremstyle[														
  spaceabove = 6pt,
  spacebelow = 6pt,
  headfont = \color{section_color}\sffamily\bfseries,
  notefont = \mdseries,
  notebraces = {(}{)},
  bodyfont = \normalfont,
  postheadspace = 1em,
  qed = ,
]{mythmstyle} %theorems
\declaretheoremstyle[														
  spaceabove = 6pt,
  spacebelow = 6pt,
  headfont = \color{section_color}\sffamily\bfseries,
  notefont = \mdseries,
  notebraces = {(}{)},
  bodyfont = \normalfont,
  postheadspace = 1em,
  qed = ,
]{myrmstyle} %remarks

\renewcommand{\labelitemi}{\color{section_color}\scriptsize$\blacksquare$} 
\renewcommand{\labelenumi}{\color{section_color}\textsf{\textbf{\arabic{enumi}}}.}

%tikz
\usepackage{tikz}
\usetikzlibrary{decorations.markings,arrows.meta}

\tikzset{
	->-/.style={decoration={
			markings,
			mark=at position .85 with {\arrow{Latex}}},postaction={decorate}},
	-<-/.style={decoration={
			markings,
			mark=at position .15 with {\arrow{Latex[reversed]}}},postaction={decorate}},
}
%tables
\usepackage{tabularx}	% tab­u­lars with ad­justable-width columns
\usepackage{multirow}	% create tab­u­lar cells span­ning mul­ti­ple rows
\usepackage{booktabs}	% publi­ca­tion qual­ity ta­bles in LaTeX
\usepackage{array}
\usepackage{dcolumn}	% align on the dec­i­mal point of num­bers in tab­u­lar columns

%hyperref
\usepackage[pdftex]{hyperref}
\hypersetup{
	pdftitle={General Relativity},
	pdfsubject={General Relativity},
	pdfkeywords={general,relativity,space,time,minkowksi},
	pdfauthor={Michael Ruf and Benjamin Rottler},
	pdfcreator={Michael Ruf and Benjamin Rottler},
	pdfproducer={Michael Ruf and Benjamin Rottler},
	bookmarksnumbered=true, % bookmarks are numbered
	bookmarksopen=true,     % show bookmarks at start of pdf viewer
	bookmarksopenlevel=2,   % level to which the bookmarks are opened
	bookmarksdepth=3,       % depth of bookmarks 
    unicode=true,           % non-Latin characters in pdf viewer's bookmarks
    pdftoolbar=false,       % show pdf viewer's toolbar?
    pdfmenubar=true,        % show pdf viewer's menu?
    pdffitwindow=false,     % window fit to page when opened
    pdfstartview={FitH},    % fits the width of the page to the window
    pdfnewwindow=true,      % links in new window
    pdfborder={0 0 1},		% no border for links
    colorlinks=false,		% false: black links; true: colored links
    linkcolor=section_color,         % color of internal links (change box color with
    % linkbordercolor)
    citecolor=green,        % color of links to bibliography
    filecolor=magenta,      % color of file links
    urlcolor=blue			% color of external links
}
\usepackage{nameref}

%general stuff
\usepackage{cite}
\usepackage{glossaries}	% create glos­saries and lists of acronyms


%%% NEW COMMANDS %%%

\newcommand\grad{\ensuremath{^\circ}}
\newcommand{\imI}{\ensuremath{\mathrm{i}}}
\newcommand{\Reals}{\ensuremath{\mathbb{R}}}
\newcommand{\Complex}{\ensuremath{\mathbb{C}}}
\newcommand{\Sphere}{\ensuremath{\mathbb{S}}}
\newcommand{\transpose}{^\top}
\newcommand{\cSym}[3]{\ensuremath{\begin{Bmatrix} #1 \\ #2 #3 \end{Bmatrix}}}
\newcommand{\csym}[3]{\ensuremath{[#1 #2,\, #3]}}
\newcommand{\affin}[3]{\ensuremath{\Gamma^{#1}_{#2 #3}}}
\newcommand{\landauO}{\mathcal{O}}
\newcommand{\name}[1]{\textsc{#1}}
\newcommand{\fourint}{\int\dif{}^4 x\,}
\DeclareMathOperator{\tr}{Tr}
\DeclareMathOperator{\re}{Re}
\DeclareMathOperator{\im}{Im}
\DeclareMathOperator{\id}{id}
\DeclareMathOperator{\Div}{div}
\let\originalleft\left  %fix bracket spacing when using \left( \right)
\let\originalright\right
\renewcommand{\left}{\mathopen{}\mathclose\bgroup\originalleft}
\renewcommand{\right}{\aftergroup\egroup\originalright}
\renewcommand{\vec}{\mathbf}
\def\mathunderline#1#2{\color{#1}\underline{{\color{black}#2}}\color{black}}
\newcommand{\liedif}[2]{\ensuremath{\mathcal{L}_{#1}#2}}
\newcommand{\lagrangian}{\mathcal{L}}
\DeclareMathOperator{\diag}{diag}
\newcommand{\const}{\text{const.}}
\DeclareMathOperator{\difD}{D}

%new environments
\newenvironment{thm}[1][]{%
  \definecolor{shadethmcolor}{rgb}{.9,.9,.95}%
  \definecolor{shaderulecolor}{rgb}{0.0,0.0,0.4}%
  %\setlength{\shadeboxrule}{1.5pt}%
  \begin{thms}[#1]\hspace*{1mm}%
}{\end{thms}}
\theoremstyle{definition}
\declaretheorem[
  style = mythmstyle,
  name = Theorem,
  shaded = {
    bgcolor = MyGray,
    padding = 2mm,
    textwidth = 0.98\textwidth
}
]{theorem}
\declaretheorem[
  style = mythmstyle,
  name = Definition,
  numberlike=theorem,
  shaded = {
    bgcolor = MyGray,
    padding = 2mm,
    textwidth = 0.98\textwidth
}
]{definition}
\declaretheorem[
  style = myrmstyle,
  name = Remark,
  numberlike=theorem,
  shaded = {
    bgcolor = white,
    padding = 2mm,
    textwidth = 0.98\textwidth
}
]{remark}
\declaretheorem[
style = myrmstyle,
name = Aside,
numberlike=theorem,
shaded = {
bgcolor = white,
padding = 2mm,
textwidth = 0.98\textwidth
}
]{sidenote}
\declaretheorem[
  style = myrmstyle,
  name = Example,
  numberlike=theorem,
  shaded = {
    bgcolor = white,
    padding = 2mm,
    textwidth = 0.98\textwidth
}
]{example}
% 
% \makeatletter
% \newtoks\FTN@ftn
% \def\pushftn{%
%  \let\@footnotetext\FTN@ftntext\let\@xfootnotenext\FTN@xftntext
%   \let\@xfootnote\FTN@xfootnote}
% \def\popftn{%
%  \global\FTN@ftn\expandafter{\expandafter}\the\FTN@ftn}
% \long\def\FTN@ftntext#1{%
%   \edef\@tempa{\the\FTN@ftn\noexpand\footnotetext
%                     [\the\csname c@\@mpfn\endcsname]}%
%   \global\FTN@ftn\expandafter{\@tempa{#1}}}%
% \long\def\FTN@xftntext[#1]#2{%
%   \global\FTN@ftn\expandafter{\the\FTN@ftn\footnotetext[#1]{#2}}}
% \def\FTN@xfootnote[#1]{%
%    \begingroup
%      \csname c@\@mpfn\endcsname #1\relax
%      \unrestored@protected@xdef\@thefnmark{\thempfn}%
%    \endgroup
%    \@footnotemark\FTN@xftntext[#1]}


%\newtheorem*{sidenote}{Sidenote}
%\newtheorem*{example}{Example}
\newenvironment{tabulars}[1]{\renewcommand*{\arraystretch}{2}\tabular{#1}}{\endtabular}		% stretched table

%%% DOCUMENT %%%

\begin{document}
\selectlanguage{english}
\onehalfspacing


\hypersetup{pageanchor=false} %stop page numbering (hyperref) to prevent for double page numers

\newcommand{\HRule}{\rule{\linewidth}{0.5mm}}
\begin{titlepage}
\begin{center}
  \HRule \\[0.4cm]
  { \huge \bfseries General Relativity}\\
  \HRule \\[0.5cm]
  \large Winter term 2015/2016 \\[0.5cm]  
  Lecture of Prof.\ Dr.\ J.\ J.\ van der Bij and Dr.\ C.\ Steinwachs\\
  Transcript of M.\ Ruf, B.\ Rottler and N.\ Luksch\\[1.5cm]
  \today
  \vfill
  \normalsize
  \name{Physikalisches Institut} \\
  \name{Albert-Ludwigs-Universität} \\
  \name{Freiburg im Breisgau}
\end{center}
\end{titlepage}

\thispagestyle{empty}
\newpage

\pagenumbering{roman}
\setcounter{page}{1}
\cfoot[- \textit{\pagemark} -]{- \textit{\pagemark} -}

\tableofcontents
\newpage

\pagenumbering{arabic} 
\hypersetup{pageanchor=true} %start page numbering again
\setcounter{page}{1}
\cfoot[\pagemark]{\pagemark}

\chapter{Newtonian Gravity}
In \name{Newton}ian physics we assume that we have absolute space and time
that can be described by the a set of numbers $x^1,x^2,x^3,t$.
We express the coordinates as functions of time.
\section{Forces}
The force $\vec{F}_{AB}$, which a massive body $A$ with mass $m_A$ exerts on another massive body $B$ with mass $m_B$, is given by
\begin{equation}
    \vec{F}_{AB}=-m_B\frac{G\textsubscript{N}m_A}{r^2}\vec{e}_r\, ,
\end{equation}
%TODO remove subscript of G
where $G\textsubscript{N}$ denotes \emph{\name{Newton}'s constant},
numerically equal to $G\textsubscript{N}\approx \unitfrac[6.673\cdot 10^{-11}]{m^3}{kg\,s}$.
Although there is no need for $G\textsubscript{N}$ to be constant over time,
there is evidence that the relative variation is less than $10^{-12}$ per year.
The force can be expressed in terms of \emph{gravitational potential} $\Phi$:
\begin{equation}
    \vec{F}_{AB}=-m_B\nabla\left(-\frac{G\textsubscript{N}m_A}{r}\right)=:
    -m_B\nabla\Phi(\vec{r}_B)\, .
\end{equation}
Given $N$ particles labeled by $n$, the total force $B$ experiences is
\begin{equation}
    \vec{F}_{B}=-\sum_n \vec{F}_{nB}\, .
\end{equation}
The potential at $\vec{r}$ is then easily found to be
\begin{equation}
    \Phi(\vec{r})=-G\textsubscript{N}\sum_n\frac{m_n}{|\vec{r}-\vec{r}_n|}\, .
\end{equation}
In general, we assume a mass distribution $\varrho(\vec{r})$ and the sum is
replaced by an integral:
\begin{equation}
    \Phi(\vec{r}) = -G\textsubscript{N}\int_{\Reals^3}\dif{\vec{r}^{\prime}}
    \frac{\varrho(\vec{r}^{\prime})}{|\vec{r}-\vec{r}^{\prime}|}\,.
\end{equation}
\section{Comparison with electrostatics}
The classical theory of gravity bears a striking similarity to electrostatics.
To make this clearer, we introduce the gravitational field
$\vec{g}(\vec{r}):= -\nabla\Phi(\vec{r})$.
\begin{table}[htb]
    \caption{Comparison of electrostatics and \name{Newton}ian gravity.}
    \begin{center}
        \begin{tabulars}{lll}
            \toprule
            &\name{Newton}ian Gravity&Electrostatics\\
            \midrule
            Force&$\displaystyle\vec{F}=q\frac{kQ}{r^2}\vec{e}_r$&$\vec{F}=m\frac{G\textsubscript{N}M}{r^2}\vec{e}_r$\\
            Potential
            &$\Phi\textsubscript{el}(\vec{r})=q\frac{kQ}{r}$
            &$\Phi\textsubscript{g}(\vec{r})=m\frac{G\textsubscript{N}M}{r}$\\
            Field
            &$\vec{E}(\vec{r})=-\nabla\Phi\textsubscript{el}(\vec{r})$
            &$\vec{g}(\vec{r})=-\nabla\Phi\textsubscript{g}(\vec{r})$\\
            \name{Laplace} equation
            &$\Delta\Phi\textsubscript{el}=-4\pi k\varrho\textsubscript{el}(\vec{r})$&
            $\Delta\Phi\textsubscript{g}=4\pi
            G\textsubscript{N}\varrho\textsubscript{g}(\vec{r})$
            \\
            \bottomrule
        \end{tabulars}
    \end{center}
\end{table}
\begin{example}[Field of a spherical mass distribution]
Assume we have a spherical mass distribution, i.e.\ $\varrho(\vec{r})=\varrho(r)$.
By symmetry considerations it follows, that the gravitational field can be
expressed by
\begin{equation}
    \vec{g}(\vec{r})=g(r)\vec{e}_r\, .
\end{equation}
We integrate the divergence of the field over a ball $B$ of radius $r$
\begin{equation}
    \int_B\dif{\vec{r}}\,\nabla\vec{g}=-\int_B\dif{\vec{r}}\, \Delta\Phi
    = -4\pi G\textsubscript{N}\int_B\dif{\vec{r}}\,\varrho(r)= -4\pi
    G\textsubscript{N} M\, ,
\end{equation}
where $M$ is the mass enclosed in $B$. On the other hand we can use Gauss's theorem to deduce
\begin{equation}
    \int_B\dif{\vec{r}}\,\nabla\vec{g}=\oint_{\partial B}\dif{\vec{A}}\cdot
    \vec{g} = \oint_{\Omega}\dif{\Omega}\, g(r)r^2=4\pi r^2g(r)\, .
\end{equation}
Together the gravitational field is given by
\begin{equation}
    \vec{g}(r)=-\frac{G\textsubscript{N}M}{r^2}\vec{e}_{r}\, .
\end{equation}
\end{example}
% \subsection{Inertial systems}
% \begin{definition}
% An inertial system is a system in which force-free particles move with constant uniform velocity on straight lines.
% \end{definition}
\subsection*{Weak Equivalence Principle (WEP)}
\name{Newton}'s first law reads
\begin{equation}
    \vec{F}=m\textsubscript{I} \vec{\ddot{x}} \, ,
\end{equation}
where $m\textsubscript{I}$ is the inertial mass that works against the acceleration of the body.
The force which a body with ``active'' mass $m\textsubscript{g,a}$ exerts on
another body with mass $m\textsubscript{g,p}$ is given by 
\begin{equation}
    \vec{F}=m\textsubscript{g,p}\frac{G\textsubscript{N}m\textsubscript{g,a}}{r^2}\vec{e}_r \, .
\end{equation}
A priori, there is no reason to assume any relation between this masses.
The first question one might ask is whether the active and the passive mass are equal.
Suppose we have two masses $A$ and $B$. Using \name{Newton}'s first law, we
can explicitly write
\begin{align}
    m^{B}_{\text{I}}\vec{\ddot{x}}&=\vec{F}_{AB}=-
    m^{B}_{\text{g,p}}\frac{G\textsubscript{N}m^{A}_{\text{g,a}}}{r^2}\vec{e}_r
    \, ,\\
    m^{A}_{\text{I}}\vec{\ddot{x}}&=\vec{F}_{BA}=-
    m^{A}_{\text{g,p}}\frac{G\textsubscript{N}m^{B}_{\text{g,a}}}{r^2}\vec{e}_r
    \, .
\end{align}
By the third law $\vec{F}_{AB}=-\vec{F}_{BA}$ we have
\begin{equation}
\frac{m^{B}_{\text{g,p}}}{m^{B}_{\text{g,a}}}=\frac{m^{B}_{\text{g,p}}}{m^{B}_{\text{g,a}}}\,.
\end{equation}
By proper choice of mass units we can set this quotient to one so that 
\begin{equation}
m\textsubscript{g,a}=m\textsubscript{g,p}=:m\textsubscript{g}
\end{equation}
The next question is weather the inertial mass equivalent to the gravitational
mass.
By \name{Newton}'s first law we have
\begin{equation}
m\textsubscript{I}\vec{\ddot{x}}
=-m_{\text{g}}\frac{G\textsubscript{N}M_{\text{g}}}{r^2}\vec{e}_r 
=-m_{\text{g}}\vec{g}\,.
\end{equation}
As a experimental result that has been measured up to a high accuracy (compare
tabular~\ref{tab:WEPExp}) all bodys recive the same acceleration due to gravity
$\ddot{\vec{x}}\sim \vec{g}$. By a 'proper' choice of units of the flight-time $t=\sqrt{\frac{m\textsubscript{I}}{m\textsubscript{g}}}\sqrt{\frac{2h}{g}}$,
we get
\begin{equation}
m\textsubscript{I}=m\textsubscript{g}=:m\,.
\end{equation}
\begin{table}
\centering
\begin{tabulars}{rllr}
\toprule
Year&Experimenter&Experiment&Accuracy\\
\midrule
1636&\name{Galilei}&inclined planes&$10^{-2}$\\
1689&\name{Newton}&pendulum&$10^{-3}$\\
1832&\name{Bessel}&pendulum&$10^{-5}$\\
1922&\name{Eötvös}&pendulum&$10^{-9}$\\
1922&\name{Shapro} et al.&pendulum&$10^{-12}$\\
1999&\name{Baesler}&torsion balance&$10^{-14}$\\
\bottomrule
\end{tabulars}
\caption{Experiments measuring the ratio
$\frac{m\textsubscript{I}}{m\textsubscript{g}}$.\label{tab:WEPExp}}
\end{table}
\subsection{Tidal Forces}
Assume we have a body of finite extension in a gravitational
potential $\Phi$, an example being the earth in the potential of the moon.
On the center of the body we have 
%TODO masses???
\begin{equation}
\dod[2]{\tensor{x}{^i}}{t}=-\dpd{\Phi}{\tensor{x}{_i}}\,.
\end{equation}
If we consider a point shifted by $\tensor{\chi}{^i}$ from the center then the
acceleration is given as
\begin{equation}
\begin{split}
\dod[2]{}{t}\left(\tensor{x}{^i}+\tensor{\chi}{^i}\right)
&=-\dpd{\Phi\left(\tensor{x}{^i}+\tensor{\chi}{^i}\right)}{\tensor{x}{_i}}\\
&\simeq-\dpd{\Phi\left(\tensor{x}{^i}\right)}{\tensor{x}{_i}}
-\dmd{\Phi\left(\tensor{x}{^i}\right)}{2}{\tensor{x}{_i}}{}{\tensor{x}{_j}}{}\tensor{\chi}{^j}\,.
\end{split}
\end{equation}
Subtracting the previous equations yields the tidal force
\begin{equation}
\dod[2]{\tensor{\chi}{^i}}{t}=-\dmd{\Phi\left(\tensor{x}{^i}\right)}{2}{\tensor{x}{_i}}{}{\tensor{x}{_j}}{}\tensor{\chi}{^j}\,.
\end{equation}
The tidal force tensor $\dmd{\Phi}{2}{\tensor{x}{_i}}{}{\tensor{x}{_j}}{}$ is of
the form
\begin{equation}
\dmd{\Phi}{2}{\tensor{x}{_i}}{}{\tensor{x}{_j}}{}
=\frac{G\textsubscript{N}M}{r^3}\left(\tensor{\delta}{_i_j}-3\frac{\tensor{x}{_i}\tensor{x}{_j}}{r^2}\right)\,.
\end{equation}
%TODO newtons laws?
%TODO picture
%TODO finish chapter 1
\chapter{Special Relativity}
\begin{table}
    \centering
    \begin{tabulars}{lll}
        \toprule
        &Newton&Einstein (SR)\\
        \midrule
        
        & Law of inertia
        & Law of inertia \\
        Newton's Laws
        & $\vec{F}=m\ddot{\vec{x}}=m\od{\vec{p}}{t}$
        &$\vec{F}=m\od{\vec{p}}{\tau}$\\
        & $\vec{F}_{12}=-\vec{F}_{21}$
        & Momentum conservations (postulate)\\
        Force
        &$\vec{F}=\frac{G\textsubscript{N}m_1m_2}{r^2}$
        &$F^\alpha= \frac{q}{m} U_\beta
        \tensor{F}{^\alpha^\beta}$,\,  $F
        =\gamma\left(\begin{smallmatrix}
        \beta
        f_0\\
        \vec{f}
        \end{smallmatrix}\right)$\\
 		&$\Delta \Phi =4\pi\rho$ &Maxwell equations
        \\
        \bottomrule
    \end{tabulars}
    \caption{Comparison of electrostatics and Newtonian gravity}
\end{table}
We want to introduce a Lagrangian for special relativity. We start by a
variation principle. We consider massive particles, i.e. timelike paths. We
postulate that the action $S$ is proportional to the spacetime distance 
\begin{equation}
S= \alpha \int_a^b\dif s=- \alpha \int_a^b\dif \tau = \alpha
\int_{t_1}^{t^2}\sqrt{1-v^2}\dif t =: \int_{t_1}^{t^2}L\dif t\, .
\end{equation}
Where the Lagrangian $L$ is given by
\begin{equation}
L=-\alpha\sqrt{1-v^2}\simeq -\alpha +\alpha\frac{1}{2}v^2+\ldots\, .
\end{equation}
We demand that we recover the classical theory in the limit $v\to 0$. The
dynamical term contribution, i.e. the lowest order term containing $v$ is 
\begin{equation}
L_0=\alpha\frac{1}{2}v^2\, .
\end{equation}
Since we know that $L_0=\frac{1}{2}mv^2$ we find $\alpha=m$. 
If we restore units of $c$ for a moment we get that the constant term in $L$ is
equal to $mc^2$. This is Einsteins famous $E=mc^2$. 

If we substitute $\alpha$ we recover the Lagrangian of special relativity
\begin{equation}
L\textsubscript{SR}=-m\sqrt{1-v^2}\, .
\end{equation}
We can proceed calculate the generalisized momenta
\begin{equation}
p_i=\dpd{L}{v_i}=\frac{mv_i}{\sqrt{1-v^2}}=\gamma m v_i\, .
\end{equation}
The energy can be calculated via the Hamiltonian $H$
\begin{equation}
E=H=\vec{p}\cdot\vec{v}-L=\gamma m v^2 + m\gamma^{-1} =\gamma m
\end{equation}
Expanded in $v$ the energy reads as
\begin{equation}
E=m+\frac{1}{2}mv^2+\ldots\, .
\end{equation}
We can further relate energy and momentum to each other. Therefore consider 
\begin{equation}
p^2=\frac{mv^2}{1-v^2}\, .
\end{equation}
Solving for $v$ yields
\begin{equation}
v^2=\frac{p^2}{p^2+m^2}\, .
\end{equation}
Which we can insert in the expression for the energy
\begin{equation}
E^2=\frac{m^2}{1-v^2}=p^2+m^2 \label{eq:onshell}\, .
\end{equation}
The equation \eqref{eq:onshell} is called the \emph{on shell condition}.
\begin{sidenote}
There is nothing in the theory that predicst that the photon is massless. It
could be possible that its mass is only really small and that in fact that the
photon does not travel at the speed of light. (This would make it
convinient to rename the constant $c$) Even though the mass should be so small
that the de Broglie wavelength is of the order of the size of the universe,
massive photons would have a huge impact.
\end{sidenote}

\chapter{Gravity and Geometry}
The observable universe is stable. There are two obvious configurations in which this is possible:
\begin{enumerate}
    \item Static universe, masses are arranged in a grid, all nett forces cancel. 
    	  However small fluctuations cause the system to collapse therefore this is
    	  no possible description for the universe. (PICTURE)
    \item Expanding universe, all masses move away from each other, overcoming the gravitational attraction. 
    	  Theoretically such a system can be described by using Newtonian Physics introducing additional energy contributions. 
    	  This turns out to be inconsistent.
\end{enumerate}
Since in the second description all particles are accelerated relative to each other, there are no inertial systems. 
A theory in which all observers are equal must therefore be local and thus be described by means of differential geometry.
Claim: The laws of physics are the same in every system.
If we assume that the Maxwell equations are right, the Newtonian theory of
gravity must be wrong.
Implications:
All free falling systems are equivalent (i.e. indistinguishable by the observer). 
Light must bend, otherwise a beam could be used to deduce whether your system is inertial.
The following example illustrates that Euclidean geometry is no adequate description of space-time.
\begin{example}[Rotationg Sphere]
see Introduction to tensor calculus
\end{example}
\section{Coordinate Systems}
We will start by studying coordinate systems in flat space, which should be
familiar.
\subsection*{Cartesian Coordinates}
(PICTURE)
\begin{equation}
    s^2=(x_1-x_2)^2+(y_1-y_2)^2
\end{equation}
Infinitesimal line element
\begin{equation}
    \dif s^2=\dif x^2+\dif y^2  \label{eq:cartline}
\end{equation}
\subsection*{Polar Coordinates}
(PICTURE)
\begin{equation}
    x= r\cos\varphi\quad y= r\sin\varphi
\end{equation}
\begin{align}
    \dif x&= \pd{x}{r}\dif r+\pd{x}{\varphi}\dif \varphi = \cos\varphi\dif r-r\sin\varphi\dif \varphi\\
    \dif y&= \pd{x}{r}\dif r+\pd{y}{\varphi}\dif \varphi = \sin\varphi\dif r+r\cos\varphi\dif \varphi
\end{align}
Plugging this into \eqref{eq:cartline} gives the line element in polar coordinates
\begin{equation}
    \dif s^2=\dif r^2+r^2\dif \varphi^2
\end{equation}
In matrix form
\begin{equation}
    \dif s^2=
    \begin{bmatrix}
        \dif r& \dif \varphi
    \end{bmatrix}
    \begin{bmatrix}
        1& 0\\
        0& r^2\\
    \end{bmatrix}
    \begin{bmatrix}
        \dif r\\ \dif \varphi
    \end{bmatrix}\, .
\end{equation}
The matrix
\begin{equation}
    g(\vec{r})=
    \begin{bmatrix}
        1& 0\\
        0& r^2\\
    \end{bmatrix}\, ,
\end{equation}
is called the \emph{metric}.
In general we have
\begin{equation}
    \dif s^2 = g_{ij}\dif x^i\dif x^j\, .
\end{equation}
The idea is to keep the law of inertia, i.e. particles still move on straight
line. However, we need to generalize the concept of a 'straight' line, in a
curved space.
Straight lines are curves minimizing the distance between two points, we therefore introduce a
\section{Variation Principle}
Again we take a look at flat space, but with curved coordinates. 
The length $S$ of a curve $\gamma$ with $\gamma^i(\lambda) = x^i(\lambda)$ is
given by the integral
\begin{equation}
    S=\int_{\gamma}\sqrt{\dif s^2} = 
    \int_{\gamma}\sqrt{g_{ij}\dif x^i\dif x^j}=\int_{a}^{b}\sqrt{g_{ij}\od{x^i}{\lambda} \od{x^j}{\lambda}}\dif \lambda\, .
\end{equation}
As stated above generalised straight lines satisfy $\delta S = 0$. It is
convinient to parametrise the curve by the arc length $\dif s$. If we define
$L:=\sqrt{g_{ij}\od{x^i}{s} \od{x^j}{s}}$, $S$ takes a form
familiar from classical mechanics:
\begin{equation}
    S=\int_a^b L\dif s\, .
\end{equation}
The extremal condition implies the Euler Lagrange equations
\begin{equation}
    \pd{}{\lambda}\pd{L}{\dot{q}_i}-\pd{L}{\dot{q}_i}
=0\, .		\end{equation}
In the case at hand the equations are
\begin{equation}
    0=\tensor{g}{_i_l}\tensor{\ddot{x}}{^j}+\frac{1}{2} \left(\partial_i g_{jl} + \partial_j g_{il} - 
    \partial_l g_{ij}\right)\dot{x}^j\dot{x}^l\, . \label{eq:PreGeo}
\end{equation}
The term invoking derivatives of the metric defines the \emph{Christoffel
symbols of the first kind}
\begin{equation}
    \csym{i}{j}{l}:=\frac{1}{2} \left(\partial_i g_{jl} + \partial_j g_{il} -
    \partial_l g_{ij}\right)\, .
\end{equation}
It is convenient to multiply \eqref{eq:PreGeo} by the inverse metric $g^{ki}$ so that we obtain the \emph{geodesic equation}
\begin{equation}
    0 = \od[2]{x^i}{s}+\cSym{i}{j}{k}\od{x^j}{s}\od{x^k}{s}\, .
\end{equation}
Where $\cSym{i}{j}{k}$ are the \emph{Christoffel symbols of the second kind}
\begin{equation}
   \cSym{i}{j}{k}:=g^{km}\csym{m}{j}{k}=\frac{1}{2}g^{il}\left(\partial_i
    g_{jl} + \partial_j g_{il} - \partial_l g_{ij}\right)\, .
\end{equation}
% remark is obsolete as long as bracket notation for christoffel symbols is used
%\begin{remark}
%Although the notation looks as the Christoffel symbols form a tensor, however
% they do not.
%\end{remark}
In flat space we have $g_{ij}=\eta_{ij}$ and can easily check that all
Christoffel symbols vanish. We therefore recover the ordinary equation of motion for a free particle
\begin{equation}
    0 = \od[2]{x^i}{s}\, .
\end{equation}
\begin{example}
Suppose a observer follows a free falling body in a homogeneous field. 
Therefore a transformation between the system of the earth and the one of the body are given by 
(for simplicity we only consider the coordinate along it is falling)
\begin{equation}
    (t,x)\to\left(t,x-\frac{1}{2}gt^2\right)
\end{equation}
Analogous to the Riemannian case discussed before, the line element takes the form
\begin{equation}
    \begin{split}
	    \dif s^2&=-\dif t^2 +\dif x^2\\
	    &=-\dif t'^2+(\dif x'- gt\dif t')(\dif x'- gt\dif t')\\
	    &=(g^2t'^2-1)\dif t'^2-2gt\dif x'\dif t'+\dif x'^2
    \end{split}
\end{equation}
\end{example}
\section{Newtonian Limit}
\chapter{Differential Geometry}
As we have noted before general relativity is a inherent local theory. It is convenient to formulate it in terms of differential geometry.
A $n$ dimensional manifold $M$ is a Hausdorff space with countable basis that is locally homeomorphic to $\mathbb{R}^n$. 
The requirements Hausdorff and countable basis are of a more technical nature and are satisfied for most of the objects one can imagine 
except some pathological examples (we won't go into the details on this).
Locally homeomorphic to $\mathbb{R}^n$ means there exists a set of \emph{charts} 
$(\varphi,U^\varphi)$ called an \emph{atlas} $\mathcal{A}$ with $\cup_{\varphi\in\mathcal{A}} U^\varphi =M$, 
i.e. the charts cover the whole manifold. The maps $\varphi:U^\varphi\to \varphi(U^\varphi)\subset\mathbb{R}^n $ are homoemorphisms, 
meaning that $U^\varphi$ is open, $\varphi$ is bijective and both $\varphi$ and $\varphi^{-1}$ are continuous.
Further for any two $\varphi,\psi\in \mathcal{A}$, the coordinate changes 
$\varphi\circ\psi^{-1}:\psi(U^\psi\cup U^\varphi)\to \phi(U^\psi\cup U^\varphi)$ be infinitely often differentiable.
(BIlder)

We can now reduce differentiation on the manifold to the ordinary differentiation in $\mathbb{R}^n$. 
Since physical laws are described in terms of differential equations, we can formulate them on $M$. 
The fact that the coordinate changes are $C^\infty$ ensures that differentiability is well defined (and thus the physical laws are).

VdB side note: There can be different \emph{differential structures} on a manifold, 
which means there are multiple differentiable structures (maximal alases) which could not be merged because the coordinate changes 
would not be $C^\infty$. Those differentiable structures therefore imply different notions of differentiability. 
Remarkably this may even play a role in some physical theories. 
As an example an 11d-supergravity can be described as a product $\mathbb{R}^{3+1}\times S^7$. 
Where $S^7$ is the 7-sphere and $\mathbb{R}^{3+1}$ Mikovski space.
This means on every point in the $\mathbb{R}^{3+1}$ there is a (small) $S^7$  located that contains additional spatial dimensions. 
Since the $S^7$ has 28 different differential structures, the choice of such a structure affects the theory for the above reasons.

All simple examples we come of can be embedded in a higher space. 
For example a 2-sphere can be interpreted as submanifold of the $\mathbb{R}^3$. 
However manifolds are objects that exists independent of such embeddings. 
For example a torus can be thought of as a square with the opposite sides identified (leaving to the left results in re-entering in the left).
VdB side note: The topological structure of the universe is a interesting question.  
On may for example imagine that we live on the surface of a 3-sphere (finite but boundless universe). 
However this might be observable in crosscorelation in the cosmic microwave background from photons reaching us 
from different directions but coming from the same event. There is no evidence of such phenomena so far. 
Most models can be excluded to some certainty, leaving only a cylindrical universe as a possible description 
(finite in one, infinite in the other directions).
\section{Vectors}
Vectors are important objects describing physics. The naive view as an "arrow pointing frow one point to another" is flawed. 
For example on a sphere an arrow connecting two points does not make much sense.
We want to find a description of vectors as objects that are naturally related to the structure of the manifold independent of the embedding.
There are three equivalent definitions for a vector.
\begin{enumerate}
    \item algebraic (mathematical, suitable for proofs)
    \item physically
    \item geometrically (ugly, but plastic)
\end{enumerate}
\paragraph{algebraic}
A vector is a derivation at the germ of a function at $p$ 
(the germ is the set of all functions that are locally equal,i.e. vectors are local objects). 
A derivation $D$ satisfies the following rules:
\begin{align*}
    D(f+g) &=Df+Dg\\
    D(\lambda)f&=\lambda f\\
    D(fg)&= (Df)g+f(Dg)
\end{align*}
Given two vectors we can construct a new one, the \emph{Lie Braket}
\begin{equation}
    [X,Y]f:=X(Yf)-Y(Xf)
\end{equation}
The only property that has to be checked is that it satisfies the Leibniz rule.
\begin{equation*}
    XY(fg)=X((Yf)g+f(Yg))=(XYf)g+(Yf)(Xg)+(Yg)(Xf)+(XYg)f\\
\end{equation*}
Subtracting $YX(fg)$ proves that $[X,Y]$ is indeed a vector.
The set of all vectors in a point $p$ is called the tangent space $T_pM$. A basis of $T_pM$ is given by $\partial_i$.
Proof sketch:
\begin{enumerate}
    \item Show $f(x^i)=f(0)+x^i\tilde{f}(x^i)$
    \item Write $X=a^i\partial_i$
    \item Show $Xf=0\quad \forall f \iff X=0$
\end{enumerate}
Every vector $A$ can be written as $A=a^i\pd{}{{x^i}}$. We can now look how the components of the vector transform 
under a change of coordinates (the vector itself is invariant!). We usually denote the elements of the transformed systems with a bar.
\begin{equation}
    A= a^k\pd{}{{x^k}}= a^k\pd{\overline{x}^i}{{x^k}}\pd{}{{\overline{x}^k}}
\end{equation}
also we can express $A$ directly in the new basis
\begin{equation}
    A= \overline{a}^i\pd{}{{\overline{x}^i}}
\end{equation}
Comparing the coefficients gives the vector transformation law
\begin{equation}
    \overline{a}^i=a^k\pd{\overline{x}^i}{{x^k}}\label{eq:coefftrafo}
\end{equation}
Sometimes a vector is defined as a object that transforms according to \ref{eq:coefftrafo} under a change of coordinates, 
this is the physical definition. It is a priori not clear that a vector also corresponds to a geometrical object. 
Consider a curve on $M$, i.e. a map $\gamma:\mathbb{R}\to M$
Then $D_\gamma f=\od{}{t}(f\circ\gamma)(0)$ is a derivative.
For the special curves $\gamma_i(t)=p+te_i$
$D_{\gamma_i} f=\partial_if$, so we can identify the derivatives with the geometrical tangent space.

Since we have a basis we can work in local coordinates, e.g. let $A=a^i\pd{}{{x^i}}$, $B=b^i\pd{}{{x^i}}$ then the lie bracket reads
\begin{equation}
    [A,B]^j=a^i\partial_ib^j-b^i\partial_ia^i
\end{equation}
Since the tangent space is a vector space, we can define its dual space
\begin{equation}
    T_pM^*=\{L:T_pM\to \mathbb{R}\, |\, L \text{ linear}\}
\end{equation}
which is again a vector space of the same dimension. Its elements are called dual or covariant vectors.
We can define a basis on $	T_pM^*$, which we denote by $\dif x^i$ and  which acts on $T_pM$ via
\begin{equation}
    \dif x^i(\partial_j)=\delta^i_j\label{eq:orthdual}
\end{equation}
It can easily deduced by \eqref{eq:orthdual} that the components of a dual vector transform as
\begin{equation}
    \overline{a}_i=\pd{x^k}{{\overline{x}^i}}a_k\, .
\end{equation}
If $\vec{a},\vec{b}\in\mathbb{R}^n$ contain the component of a vector and a dual vector respectively, 
then the transformation can be written in matrix form
\begin{align*}
    \vec{a}&\to V\vec{a}\\
    \vec{b}&\to\left(V^T\right)^{-1}\vec{b}
\end{align*}
with $V_{ij}=\pd{\overline{x}^i}{{x^j}}$. 
In normal calculus we restrict ourself to orthogonal transformations (i.e. mapping orthonormal bases onto each other) for which $(O^T)^{-1}=O$. 
Which is the reason why we do not bother to distinguish between vectors and dual vectors because they transform identically. 
In special relativity we have e.g. $(\Lambda^T)^{-1}\neq\Lambda$ for a boost, the difference becomes even more important in general 
relativity where the relation can become arbitrarily complicated.
\section{Tensors}
From vectors $A$ ,$B$ we can construct new objects with multiple indices that posses well defined transformation behaviours. 
For example we can define
\begin{equation}
    \overline{T}^{ij}=a^ib^j
\end{equation}
Which transforms as
\begin{equation}
    T^{ij}=\pd{\overline{x}^i}{{x^k}}\pd{\overline{x}^j}{{x^l}}a^kb^l=\pd{\overline{x}^i}{{x^k}}\pd{\overline{x}^j}{{x^l}}T^{kl}\label{eq:tensortrafo}
\end{equation}
Again it is possible to define tensors in a coordinate independent way. 
At this point we will make things easier and only consider the physical definition. 
A tensor is then a object that transforms similar to \eqref{eq:tensortrafo}.
\paragraph{Symmetries}
A tensor is said to be symmetric in two indices if it stays invariant when exchanging those indices, e.g.
\begin{equation}
    T_{ab}=T_{ba}
\end{equation}
Remark: We have not yet established a relation between upper and lower indices, i.e. we have no metric. Expressions of the form
\begin{equation}
    T_a^b=T_b^a
\end{equation}
make no sense since they can not be true in every system.
\section{The Metric}
The metric $g$ is a non degenerate ($\det(g)\neq 0$), symmetric covariant two tensor. 
We have already seen examples of metrics for the flat space, e.g. in spherical coordinates $g$ was given as
\begin{equation}
    g=
    \begin{pmatrix}
        1 & 0\\
        0 & r^2\\
    \end{pmatrix}
\end{equation}
Given a metric we relate vectors and dual vectors to each other by
\begin{equation}
    a_i=g_{ij}a^j
\end{equation}
\section{Parallel Transport}
vdB side note: Example from Electrodynamics concerning the covariant derivative. 
The theory is invariant under transformations $\phi\to e^{\imI \alpha}\phi$, 
because $\phi^*\phi$ and\\ $\phi^*\nabla\phi-\phi\nabla\phi^*$do not change.

\chapter{Einstein's Field Equations}
\chapter{The Energy Momentum Tensor}
In special relativity we have seen that the energy momentum tensor
$\tensor{T}{^\mu^\nu}$ is conserved or divergence free respectively, i.e.
\begin{equation}
\tensor{\partial}{_\mu}\tensor{T}{^\mu^\nu}=0\,.\label{eq:EMcons}
\end{equation}
The problem we are faced in general relativity is the very definition of local
energy. Because gravity surely contributes to the energy, a problem arises
as we can always transform to local flat space.
We start by revisiting the example of dust
\begin{example}[Dust]
\begin{equation}
\tensor{T}{^\mu^\nu}=\rho_0\tensor{u}{^\mu}\tensor{u}{^\nu}\,.
\end{equation}
In special relativity:
${\tensor{u}{^\mu}=\od{\tensor{x}{^\mu}}{\tau}=\gamma(1,\vec{v})\transpose}$,
${\tensor{T}{^0^0}=\rho_0\left(\od{t}{\tau}\right)^2}=\gamma^2\rho_0:=\rho$.
Where $\rho$ is the density with respect to an observer at rest. For the Volume
we have $V=\gamma^{-1}V_0$. For the Energy $E=\gamma\omega_0$. Then the density
is given by $\rho=\frac{E}{V}=\gamma^2\rho_0$. The conservation of
$\tensor{T}{^0^\nu}$ implies
\begin{equation}
\begin{split}
0&=\tensor{T}{^0^\nu_{,\nu}}\\
&=\tensor{T}{^0^0_{,0}}+\tensor{T}{^0^i_{,i}}\\
&=\tensor{\partial}{_t}\left(\rho_0\gamma^2\right)
+\tensor{\partial}{_i}\left(\rho_0\gamma^2\tensor{v}{^i}\right)\\
&=\tensor{\partial}{_t}\rho
+\tensor{\partial}{_i}\left(\rho \tensor{v}{^i}\right)\\
&=\dot{\rho}
+\boldsymbol{\nabla}\left(\rho \vec{v}\right)\,,
\end{split}
\end{equation}
the \emph{continuity equation}. For the remaining spatial components we get
\begin{equation}
\begin{split}
0&=\tensor{T}{^i^\nu_{,\nu}}\\
&=\tensor{T}{^i^0_{,0}}+\tensor{T}{^j^i_{,i}}\\
&=\dot{\rho}\tensor{v}{^i}
+\rho\tensor{\dot{v}}{^i}
+\tensor{v}{^i}\tensor{\partial}{_j}\left(\rho\tensor{v}{^j}\right)
+\tensor{\dot{v}}{^j}\rho\tensor{\partial}{_j}\tensor{v}{^i}\\
&=\tensor{v}{^i}\left[\dot{\rho}+\tensor{\partial}{_j}\left(\rho\tensor{\dot{v}}{^j}\right)\right]
+\rho\left(\tensor{\dot{v}}{^i}+\tensor{\dot{v}}{^j}\tensor{\partial}{_j}\tensor{v}{^i}\right)\\
&=\rho\left(\tensor{\dot{v}}{^i}+\tensor{\dot{v}}{^j}\tensor{\partial}{_j}\tensor{v}{^i}\right)\,,
\end{split}
\end{equation}
the \emph{Euler equation} for vanishing pressure (which was the key assumption
for dust). It is natural to generalize equation \eqref{eq:EMcons} to curved
space
\begin{equation}
\tensor{T}{^\mu^\nu_{;\nu}}=0\,.
\end{equation}
In expanded form
\begin{equation}
\begin{split}
0
&=\tensor{\nabla}{_\nu}\left(\rho_0\tensor{u}{^\mu}\tensor{u}{^\nu}\right)\\
&=\tensor{{\rho_0}}{_{;\nu}}\tensor{u}{^\mu}\tensor{u}{^\nu}
+\rho_0\tensor{u}{^\mu_{;\nu}}\tensor{u}{^\nu}
+\rho_0\tensor{u}{^\mu}\tensor{u}{^\nu_{;\nu}}\\
&=\tensor{u}{^\mu}\left(\rho_0\tensor{u}{^\nu_{;\nu}}+\tensor{{\rho_0}}{_{;\nu}}\tensor{u}{^\nu}\right)
+\rho_0\tensor{u}{^\mu_{;\nu}}\tensor{u}{^\nu} \label{eq:DustEnCons}
\end{split}
\end{equation}
We multiply both sides with $\tensor{u}{_\mu}$
\begin{equation}
\begin{split}
0
&=-\left(\rho_0\tensor{u}{^\nu_{;\nu}}+\tensor{{\rho_0}}{_{;\nu}}\tensor{u}{^\nu}\right)
+\rho_0\tensor{u}{^\nu}\tensor{u}{_\mu}\tensor{u}{^\mu_{;\nu}}\\
&=-\left(\rho_0\tensor{u}{^\nu_{;\nu}}
+\tensor{{\rho_0}}{_{;\nu}}\tensor{u}{^\nu}\right)
\end{split}
\end{equation}
If we plugg this back into equation \eqref{eq:DustEnCons} we get
\begin{equation}
\begin{split}
0&=\tensor{u}{^\nu}\tensor{\nabla}{_\nu}\tensor{u}{^\mu}\\
&=\tensor{u}{^\nu}\tensor{\partial}{_\nu}\tensor{u}{^\mu}
+\tensor{u}{^\nu}\cSym{\mu}{\nu}{\rho}\tensor{u}{^\rho}\\
&=\dod{\tensor{x}{^\nu}}{\tau}\dpd{\tensor{u}{^\mu}}{\tensor{x}{^\nu}}
+\cSym{\mu}{\nu}{\rho}\dod{\tensor{x}{^\nu}}{\tau}\dod{\tensor{x}{^\nu}}{\tau}\\
&=\dpd[2]{\tensor{x}{^\mu}}{\tau}
+\cSym{\mu}{\nu}{\rho}\dod{\tensor{x}{^\nu}}{\nu}\dod{\tensor{x}{^\nu}}{\tau}\\
\end{split}
\end{equation}
The geodesic equation \eqref{eq:geodeq}
\end{example}
\begin{remark}
This is a difference between electrodynamics and general relativity; dust moves
on geodesics, i.e. the path is determined by the field equations alone. In contrast
in electrodynamics an additional Force (Lorentz force) has to be
\emph{postulated} to describe the motion of test particles. The case is not
settled however e.g. it is unclear whether the paths of spin particles is also
determined by the field equations.
\end{remark}
\chapter{Linearized theory and Newtonian limit}
\section{Linearized theory}
Consider a weak gravitational field. Then we can split the full spacetime metric $g_{\mu\nu}(x)$ into two parts.
\begin{definition}[Linearization of the metric field.]
    \begin{equation}
        g_{\mu\nu}(x) = \eta_{\mu\nu} + h_{\mu\nu}(x) + \landauO(h^2) \, .
    \end{equation}
\end{definition}
$\eta_{\mu\nu}$ is the flat, constant ``background'' metric of Minkowski space, i.e.\ there is no gravitational field present.
The field $h_{\mu\nu}(x)$ can be interpreted as a perturbation on the fixed background $\eta_{\mu\nu}$.
One can identify a spin-2 particle, the so-called \emph{graviton}, with the excitations (quantized fluctuations) of this field.
Because only the linear order of $h$ is considered, the nonlinearity of Einstein's equations is lost.
We can raise and lower indices with $\eta_{\mu\nu}$ and $\eta^{\mu\nu}$.

\begin{remark}
This works only for a weak gravitational field, since a strong gravitational field produces a strong back reaction of ``matter''
on the geometry, which follows from the nonlinearity of Einstein's equations.
Exactly this back reaction is neglected in the linearized theory.
\end{remark}

\subsection{Derivation of the linearized Einstein's equations}
In the following we neglect all terms with $\landauO(h^2)$.
Our goal is to express Einstein's field equations in the linearized approximation.
For this we need to calculate the Christoffel symbols, the Riemann tensor, the Ricci tensor and the Ricci scalar.

Christoffel symbols of the first kind:
\begin{equation}
    \csym{\mu}{\nu}{\varrho} = \frac{1}{2} \left( \tensor{h}{_\mu_\varrho_,_\nu} + \tensor{h}{_\nu_\varrho_,_\mu}
    - \tensor{h}{_\mu_\nu_,_\varrho} \right) + \landauO(h^2) \, .
\end{equation}
Christoffel symbols of the second kind:
\begin{equation}
    \cSym{\varrho}{\mu}{\nu} = g^{\varrho\sigma} \csym{\mu}{\nu}{\sigma} = \eta^{\varrho\sigma} \csym{\mu}{\nu}{\sigma} + \landauO(h^2)
    = \frac{1}{2} \left( \tensor{h}{_\mu^\varrho_,_\nu} + \tensor{h}{_\nu^\varrho_,_\mu} - \tensor{h}{_\mu_\nu^,^\varrho} \right) + \landauO(h^2) \, .
\end{equation}
Riemann tensor:
\begin{equation}
    \begin{split}
        \tensor{R}{^\varrho_\sigma_\mu_\nu}
        &= \partial_\mu \cSym{\varrho}{\nu}{\sigma} - \partial_\nu \cSym{\varrho}{\mu}{\sigma}
        + \underbrace{\cSym{\varrho}{\mu}{\lambda} \cSym{\lambda}{\nu}{\sigma} - \cSym{\varrho}{\nu}{\lambda} \cSym{\lambda}{\mu}{\sigma}}_{\landauO(h^2)} \\
        &= \frac{1}{2} \left( \tensor{h}{_\nu^\varrho_,_\sigma_\mu} + \mathunderline{blue}{\tensor{h}{^\varrho_\sigma_,_\nu_\mu}} - \tensor{h}{_\nu_\sigma^,^\varrho_\mu}
        - \mathunderline{blue}{\tensor{h}{^\varrho_\mu_,_\sigma_\nu}} + \tensor{h}{^\varrho_\sigma_,_\mu_\nu} + \tensor{h}{_\mu_\sigma^,^\varrho_\nu} \right) + \landauO(h^2) \\
        &= \frac{1}{2} \left( \tensor{h}{_\nu^\varrho_,_\sigma_\mu} - \tensor{h}{_\nu_\sigma^,^\varrho_\mu}
        - \tensor{h}{^\varrho_\sigma_,_\mu_\nu} + \tensor{h}{_\mu_\sigma^,^\varrho_\nu} \right) + \landauO(h^2) \, .
    \end{split}
\end{equation}
Ricci tensor
\begin{equation}
    \tensor{R}{_\sigma_\nu} = \tensor{R}{^\varrho_\sigma_\varrho_\nu}
    = \frac{1}{2} \left( \tensor{h}{_\nu^\varrho_,_\sigma_\varrho} - \tensor{h}{_\nu_\sigma^,^\varrho_\varrho}
    - \tensor{h}{_,_\sigma_\nu} + \tensor{h}{_\varrho_\sigma^,^\varrho_\nu} \right) + \landauO(h^2)\, ,
\end{equation}
with $h\coloneqq h_{\mu\nu}\eta^{\mu\nu}$

Ricci scalar
\begin{equation}
    R = g^{\sigma\nu}R_{\sigma\nu} = \eta^{\sigma\nu}R_{\sigma\nu} + \landauO(h^2)
    = \tensor{h}{^\sigma^\nu_,_\sigma_\nu} - \tensor{h}{_,_\sigma^\sigma} + \landauO(h^2)
\end{equation}
We define
\begin{equation}
    \overline{h}_{\mu\nu} \coloneqq h_{\mu\nu} - \frac{1}{2} \eta_{\mu_\nu}h\,.
\end{equation}
With the following two lines we can show that $\overline{\overline{h}} = h$:
\begin{align}
    \overline{h} \coloneqq &\ \overline{h}_{\mu\nu}\eta^{\mu\nu} = h - 2h = -h \\
    h_{\mu_\nu} =&\ \overline{h}_{\mu\nu} + \frac{1}{2} \eta_{\mu\nu}h = \overline{h}_{\mu\nu} - \frac{1}{2} \eta_{\mu\nu}\overline{h}
\end{align}

\subsubsection{Linearized Einstein tensor \texorpdfstring{$G_{\mu\nu}$}{Gmunu} in terms of \texorpdfstring{$\overline{h}_{\mu\nu}$}{hbarmunu}}

\begin{equation}
    \begin{split}
        G_{\mu\nu}^{\text{(L)}} =\ & R_{\mu\nu}^{\text{(L)}} - \frac{1}{2} \eta_{\mu\nu} R^{\text{(L)}} \\
        =\ & \frac{1}{2} \partial_\mu \partial_\varrho \tensor{h}{_\nu^\varrho} + \frac{1}{2} \partial_\nu \partial_\varrho \tensor{h}{_\mu^\varrho}
        - \frac{1}{2} \Box h_{\mu\nu} - \frac{1}{2} \partial_{\mu\nu}h-\frac{1}{2} \eta_{\mu\nu}\partial_\varrho\partial_\sigma h^{\varrho\sigma}
        + \frac{1}{2} \eta_{\mu_\nu}\Box h \\
        =\ & \frac{1}{2} \partial_\mu\partial_\varrho \tensor{\overline{h}}{_\nu^\varrho}
        - \mathunderline{blue}{\frac{1}{4}\partial_\mu\partial_\nu\overline{h}}
        + \frac{1}{2} \partial_\nu\partial_\varrho\tensor{\overline{h}}{_\mu^\varrho}
        - \mathunderline{blue}{\frac{1}{4}\partial_\nu\partial_\mu\overline{h}} - \frac{1}{2}\Box\overline{h}_{\mu\nu} \\
        & + \mathunderline{green}{\frac{1}{2}\eta_{\mu\nu}\Box\overline{h}} + \mathunderline{blue}{\frac{1}{2}\partial_\mu\partial_\nu\overline{h}}
        - \frac{1}{2}\eta_{\mu\nu}\partial_\varrho\partial_\sigma\overline{h}^{\varrho\sigma}
        + \mathunderline{green}{\frac{1}{4}\eta_{\mu\nu}\Box\overline{h}} - \mathunderline{green}{\frac{1}{2}\eta_{\mu\nu}\Box\overline{h}} \\
        =\ & -\frac{1}{2} \Box \overline{h}_{\mu\nu} + \partial_\varrho \tensor{\partial}{_(_\mu}\tensor{\overline{h}}{_\nu_)^\varrho}
        - \frac{1}{2} \eta_{\mu\nu}\partial_\varrho\partial_\sigma \overline{h}^{\varrho\sigma} \\
        \overset{!}{=}\ & \kappa T_{\mu\nu}
    \end{split}
\end{equation} %TODO introduce symmetration brackets somewhere
with the (linearized) d'Alembert operator
\begin{equation}
    \Box^{\text{(L)}} = \Box = \partial_\mu\partial_\nu \eta^{\mu\nu}=\partial_\mu\partial^\mu
\end{equation}
\begin{definition}[Linearized Einstein equations]
    \begin{equation}
        \label{eq:lineinsteineqs}
        -\frac{1}{2} \Box \overline{h}_{\mu\nu} + \partial_\varrho \tensor{\partial}{_(_\mu}\tensor{\overline{h}}{_\nu_)^\varrho}
        - \frac{1}{2} \eta_{\mu\nu}\partial_\varrho\partial_\sigma \overline{h}^{\varrho\sigma} = \kappa T_{\mu\nu}
    \end{equation}
\end{definition}

\subsubsection{Gauge transformations}
Usually field equations are in the form of
\begin{equation}
    \Box \text{``field''} = \text{``source''}
\end{equation}
Equation~\eqref{eq:lineinsteineqs} can be written in this form:
\begin{equation}
    \underbrace{\Box \overline{h}_{\mu\nu}}_{\Box\text{``field''}}
    \underbrace{- 2 \partial_\varrho \tensor{\partial}{_(_\mu} \tensor{\overline{h}}{_\nu_)^\varrho}
    + \eta_{\mu\nu}\partial_\varrho \overline{h}^{\varrho\sigma}}_{\text{ensures gauge invariance of equation}}
    = \underbrace{-2\kappa T_{\mu\nu}}_{\text{``source''}}
\end{equation}

TODO Michi: mathematical background \\%TODO mathematical background @Michi
Infinitesimal diffeomorphisms = affine transformations
\begin{definition}
    \begin{equation}
        x^\mu = x'^\mu + \xi^\mu(x'^\mu), \qquad \xi^\mu \ll 1
    \end{equation}
\end{definition}
In the following we neglect terms with $\landauO(\xi^2)$, $\landauO(\xi h)$, and $\landauO(h^2)$ and higher order terms.
The transformed metric reads
\begin{equation}
    \begin{split}
        \eta_{\mu\nu} + h'_{\mu\nu}(x') &= g'_{\mu\nu} \\
        &= \frac{\partial x^\varrho}{\partial x'^\mu} \frac{\partial x^\sigma}{\partial x'^\nu} g_{\varrho\sigma}(x) \\
        &= \frac{\partial \left( x'^\varrho + \xi^\varrho \right)}{\partial x'^\mu}
        \frac{\partial \left( x'^\sigma + \xi^\sigma \right)}{\partial x'^\nu}
        \left( \eta_{\varrho\sigma} + h_{\varrho\sigma}(x) \right) + \landauO(\xi^2) \\
        &= \left( \tensor{\delta}{_\mu^\varrho} + \tensor{\xi}{^\varrho_,_\mu} \right)
        \left( \tensor{\delta}{_\nu^\sigma} + \tensor{\xi}{^\sigma_,_\nu} \right)
        \left( \eta_{\varrho\sigma} + h_{\varrho\sigma}(x) \right) + \landauO(\xi^2) \\
        &= \left( \tensor{\delta}{_\mu^\varrho} + \tensor{\xi}{^\varrho_,_\mu} \right)
        \left( \eta_{\varrho\nu} + h_{\varrho\nu} + \tensor{\xi}{_\varrho_,_\nu} \right) + \landauO(\xi^2)  \\
        &= \eta_{\mu\nu} + h_{\mu\nu} + \tensor{\xi}{_\mu_,_\nu} + \tensor{\xi}{_\nu_,_\mu} + \landauO(\xi^2,h^2,\xi h)
    \end{split}
\end{equation}
$\implies$ The perturbation $h_{\mu\nu}$ transforms under infinitesimal diffeomorphisms in the following way
\begin{equation}
    \begin{split}
        h'_{\mu\nu}(x) &= h_{\mu\nu}(x) + \tensor{\xi}{_\mu_,_\nu} + \tensor{\xi}{_\nu_,_\mu} \\
        &= h_{\mu\nu}(x) + \left(\liedif{\xi}{\eta} \right)_{\mu\nu}
    \end{split}
\end{equation}
\begin{definition}[Lie derivative]
    The Lie derivative off a tensor field $T$ with $k$ contravariant and $l$ covariant indices along the vector $\xi$ is given by
    \begin{equation}
        \begin{split}
            \left( \liedif{\xi}{T} \right)^{\alpha_1\ldots\alpha_k}_{\beta_1\ldots\beta_l}
            \coloneqq \xi^\mu \partial_\mu T^{\alpha_1\ldots\alpha_k}_{\beta_1\ldots\beta_l}
            & - \left( \partial_\mu \xi^{\alpha_1} \right) T^{\mu\alpha_2\ldots\alpha_k}_{\beta_1\ldots\beta_l} - \ldots
            - \left( \partial_\mu \xi^{\alpha_k} \right) T^{\alpha_1\ldots\alpha_{k-1}\mu}_{\beta_1\ldots\beta_l} \\
            & + \left( \partial_{\beta_1} \xi^\mu \right) T^{\alpha_1\ldots\alpha_k}_{\mu\beta_2\ldots\beta_l} + \ldots
            + + \left( \partial_{\beta_l} \xi^\mu \right) T^{\alpha_1\ldots\alpha_k}_{\beta_1\ldots\beta_{l-1}\mu}
        \end{split}
    \end{equation}
\end{definition}
Therefore
\begin{equation}
    \left( \liedif{\xi}{\eta} \right)_{\mu_\nu} = \underbrace{\xi^\varrho\partial_\varrho\eta_{\mu\nu}}_{=0}
    + \tensor{\xi}{_\mu_,_\nu} + \tensor{\xi}{_\nu_,_\mu} = \tensor{\xi}{_\mu_,_\nu} + \tensor{\xi}{_\nu_,_\mu}
\end{equation}
If the derivative of a metric vanishes for a given $\xi^\mu$, then one obtains the killing equations for $\xi^\mu$ and the solution of the
killing equations are denoted killing vector field.
\begin{equation}
    \tensor{\xi}{_\mu_,_\nu} + \tensor{\xi}{_\nu_,_\mu}=0  %TODO order of indices in second term?
\end{equation}
In Minkowski-space the ten infinitesimal killing vectors correspond to the Poincaré-generators.
\begin{sidenote}
Lie-derivative on metric $\implies$ detect symmetries of Manifold.
\end{sidenote}
\subsubsection{Check that linearized field equations are invariant under infinitesimal diffeomorphisms}
\begin{equation}
    h'_{\mu\nu} = h_{\mu\nu} + \tensor{\xi}{_\mu_,_\nu} + \tensor{\xi}{_\nu_,_\mu}
\end{equation}
\begin{equation}
    \begin{split}
        \overline{h'}_{\mu\nu} &= h'_{\mu\nu} - \frac{1}{2} \eta_{\mu\nu}h' \\
        &= h_{\mu\nu} + \tensor{\xi}{_\mu_,_\nu} + \tensor{\xi}{_\nu_,_\mu} - \frac{1}{2} \eta_{\mu\nu}h
        -\frac{1}{2}\eta_{\mu\nu}\partial^\varrho\xi_\varrho - \frac{1}{2} \eta_{\mu\nu}\partial^\varrho\xi_\varrho \\
        &= \overline{h}_{\mu\nu} + \tensor{\xi}{_\mu_,_\nu} + \tensor{\xi}{_\nu_,_\mu} - \eta_{\mu\nu}\tensor{\xi}{^\varrho_,_\varrho}
    \end{split}
\end{equation}
Plug this into Einstein's equations (term by term)
\begin{align}
    -\frac{1}{2}\Box \overline{h'}_{\mu\nu} &= -\frac{1}{2}\Box\overline{h}_{\mu\nu} - \frac{1}{2}\Box\tensor{\xi}{_\mu_,_\nu}
    -\frac{1}{2}\Box\tensor{\xi}{_\nu_,_\mu} + \frac{1}{2} \eta_{\mu\nu}\Box\tensor{\xi}{^\varrho_,_\varrho} \\
    -\frac{1}{2} \eta_{\mu\nu}\partial_\varrho\partial_\sigma \overline{h'}^{\varrho\sigma} &=
    -\frac{1}{2} \eta_{\mu\nu}\partial_\varrho\partial_\sigma \left( \tensor{\xi}{^\varrho^,^\sigma} + \tensor{\xi}{^\sigma^,^\varrho}
    - \eta^{\varrho\sigma} \tensor{\xi}{^\alpha_,_\alpha} \right) - \frac{1}{2} \eta_{\mu\nu}\partial_\varrho\partial_\sigma \overline{h}^{\varrho\sigma} \\
    \partial^\varrho \tensor{\partial}{_(_\mu} \tensor{\overline{h'}}{_\nu_)_\varrho} &=
    \partial^\varrho \tensor{\partial}{_(_\mu} \tensor{\overline{h}}{_\nu_)_\varrho} + \frac{1}{2}\Box\tensor{\xi}{_\nu_,_\mu}
    + \frac{1}{2}\Box\eta_{\mu\nu}
\end{align}
Therefore
\begin{equation}
    \begin{split}
        & -\frac{1}{2} \Box \overline{h'}_{\mu\nu} + \partial_\varrho \tensor{\partial}{_(_\mu}\tensor{\overline{h'}}{_\nu_)^\varrho}
        - \frac{1}{2} \eta_{\mu\nu}\partial_\varrho\partial_\sigma \overline{h'}^{\varrho\sigma} \\
        =\ & -\frac{1}{2}\Box\overline{h}_{\mu\nu} - \mathunderline{blue}{\frac{1}{2}\Box\tensor{\xi}{_\mu_,_\nu}}
        -\mathunderline{blue}{\frac{1}{2}\Box\tensor{\xi}{_\nu_,_\mu}}
        + \mathunderline{green}{\frac{1}{2} \eta_{\mu\nu}\Box\tensor{\xi}{^\varrho_,_\varrho}}
        + \partial^\varrho \tensor{\partial}{_(_\mu}\tensor{\overline{h}}{_\nu_)_\varrho} \\
        & + \mathunderline{blue}{\Box\tensor{\xi}{_(_\mu_,_\nu_)}}
        - \mathunderline{green}{\frac{1}{2}\eta_{\mu\nu}\Box\tensor{\xi}{^\varrho_,_\varrho}}
        - \frac{1}{2} \eta_{\mu\nu}\partial_\varrho\partial_\sigma \overline{h}^{\varrho\sigma} \\
        =\ & -\frac{1}{2} \Box \overline{h}_{\mu\nu} + \partial_\varrho \tensor{\partial}{_(_\mu}\tensor{\overline{h}}{_\nu_)^\varrho}
        - \frac{1}{2} \eta_{\mu\nu}\partial_\varrho\partial_\sigma \overline{h}^{\varrho\sigma}
    \end{split}
\end{equation}
This shows that the Einstein equations are invariant under an infinitesimal diffeomorphisms.
Therefore $\overline{h}_{\mu\nu}$ and $\overline{h'}_{\mu\nu}$ are the same \emph{physical} field.

\subsubsection{Harmonic gauge in linearized gravity}
We would like to have the linearized field equations in the form $\Box\text{``field''}=\text{``source''}$ (wave equation).
This can ge done with the gauge condition
\begin{equation}
    \chi_\nu \left[ \overline{h} \right] \coloneqq \partial^\mu \overline{h}_{\mu\nu} = 0.
\end{equation}
In therms of the original field this condition reads
\begin{definition}[de Donder gauge, harmonic gauge]
    \begin{equation}
        \chi_\nu \left[ h \right] = \partial^\mu h_{\mu\nu} - \frac{1}{2} h_{\mu\nu} \partial^\mu h = 0
    \end{equation}
\end{definition}
Proof:
\begin{equation}
    \begin{split}
        \partial^\mu \overline{h'}_{\mu\nu} &= \partial^\mu \overline{h}_{\mu\nu} + \Box \xi_\nu + \partial_\nu \partial^\mu \xi_\mu -
        \eta_{\mu\nu} \partial^\mu\partial_\varrho\xi^\varrho \\
        &= \partial^\mu \overline{h}_{\mu\nu} + \Box \xi_\nu = 0
    \end{split}
\end{equation}
Solve for $\Box\xi_\nu$
\begin{equation}
    \implies \Box \overline{h'}_{\mu\nu} = -2\kappa T_{\mu\nu}
\end{equation}
Since $\overline{h}_{\mu\nu}$ and $\overline{h'}_{\mu\nu}$ correspond to the same physical field configuration, we can drop the prime.
\begin{definition}{Linearized field equations in de Donder gauge.}
    \begin{equation}
        \Box \overline{h}_{\mu\nu} = - 2 \kappa T_{\mu\nu}
    \end{equation}
\end{definition}
\chapter{Gravitational Waves}
\begin{itemize}
    \item vacuum solution of linearized Einstein equations
    \begin{equation}
        \Box \overline{h}_{\mu\nu} = 0 \qquad \text{in de Donder gauge}
    \end{equation}
    \item describes \emph{weak} gravitational waves \emph{only} $\rightarrow$ linearized treatment justified
    \item description breaks down for strong gravitational fields as the theory becomes essentially \emph{non-linear} (e.g.\ two black holes merge)
    \item analysis similar to electrodynamics, but here $h_{\mu\nu}$: spin-2 field, $A_\mu$: spin-1 field
    \item vacuum equations
    \begin{equation}
        \Box \overline{h}_{\mu\nu} = 0 \qquad \overline{h}_{\mu\nu}=h_{\mu\nu} - \frac{1}{2} \eta_{\mu\nu} h \qquad \partial^\mu \overline{h}_{\mu\nu} = 0
    \end{equation}
    \item Gauge freedom not yet completely exhausted by de Donder gauge.
    \begin{equation}
        \partial^\mu \overline{h'}_{\mu\nu} = \underbrace{\partial^\mu \overline{h'}_{\mu\nu}}_{=0} + \Box \xi_\nu = 0
    \end{equation}
    $\implies$ All gauge transformations generated by $\xi_\nu$ that satisfy $\Box \xi_\nu = 0$ do not lead out of the de Donder gauge. \\
    Compare with electromagnetic field:
    \begin{align}
        & A_\mu \to {A'}^\mu = A^\mu + \partial_\mu \lambda(x) \\
        & \partial_\mu A^\mu = 0 \qquad \text{Lorentz gauge} \\
        & \partial_\mu {A'}^\mu = \underbrace{ \partial_\mu A^\mu}_{=0} + \partial_\mu \partial^\mu \lambda(x) = 0 \implies \Box \lambda = 0
    \end{align}
    Exploit this remaining gauge freedom to make perturbations $h_{\mu\nu}$ \emph{transverse} and \emph{traceless}
    \item transversality
    \begin{equation}
        \partial^\mu h'_{\mu\nu} = \partial^\mu h_{\mu\nu} + \Box \xi_\nu + \partial_\nu \partial^\mu \xi_\mu \overset{!}{=} 0
    \end{equation}
    Since only gauge transformations that satisfy $\Box \xi_\nu$ are allowed (do not lead out of de Donder gauge), the equation that shoud
    be solved for $\xi_\nu$ is
    \begin{equation}
        \partial_\nu \partial^\mu \xi_\mu = - \partial^\mu h_{\mu\nu}
    \end{equation}
    \item In order for the perturbations to be traceless, we need to find a solution to
    \begin{equation}
        h' = h + \partial_\mu \xi^\mu = 0 \implies \partial_\mu \xi^\mu =
        -\frac{1}{2}h
    \end{equation}
\end{itemize}
\begin{figure}[hbtp!]
\centering
 \includegraphics{gwave1.pdf}
  \includegraphics{gwave2.pdf}
  \includegraphics{gwave3.pdf}
\caption{Polarisations of gravitational waves. From top to bottom:
$+$,$\times$ and mixed (circular) polarised waves.}
%TODO Caption
\end{figure}

% \begin{figure}
% \centering
% \begin{tikzpicture}[decoration={markings,
%   mark=between positions 0 and 1 step 22.3pt
%   with { \draw [fill] (0,0) circle [radius=2pt];}}]
% \draw[->] (0,1) -- (3,1);
% \draw[->] (0,1) -- (0,4);
% \node[text width=0.25cm] at (3,1.25){$x$};
% \node[text width=0.25cm] at (0.25,4){$y$};
%
% \node[text width=2cm, align=center] (1) at (2,5){$\omega t= 0$};
% \node[text width=2cm, align=center] (1) at (5,5){$\omega t= \frac{\pi}{2}$};
% \node[text width=2cm, align=center] (1) at (8,5){$\omega t= \pi$};
% \node[text width=2cm, align=center] (1) at (11,5){$\omega t= \frac{3\pi}{2}$};
% \node[text width=2cm, align=center] (1) at (14,5){$\omega t= 2\pi$};
% \begin{scope}[]
% \draw[dots along my path]  (2,3) ellipse (1 and 1);
% \end{scope}
% \begin{scope}[shift={(3,-0.5)},transform canvas={yscale=1.2}]
% \draw[dots along my path]  (2,3) ellipse (1 and 1);
% \end{scope}
% \begin{scope}[shift={(6,0)}]
% \draw[dots along my path]  (2,3) ellipse (1 and 1);
% \end{scope}
% \begin{scope}[shift={(7.2,0)},transform canvas={xscale=1.2}]
% \draw[dots along my path]  (2,3) ellipse (1 and 1);
% \end{scope}
% \begin{scope}[shift={(12,0)}]
% \draw[dots along my path]  (2,3) ellipse (1 and 1);
% \end{scope}
% \end{tikzpicture}
% \caption{Gravitational wave.}
% \end{figure}
% \begin{figure}
% \centering
% \begin{tikzpicture}[decoration={markings,
%   mark=between positions 0 and 1 step 22.3pt
%   with { \draw [fill] (0,0) circle [radius=2pt];}}]
% \draw[->] (0,1) -- (3,1);
% \draw[->] (0,1) -- (0,4);
% \node[text width=0.25cm] at (3,1.25){$x$};
% \node[text width=0.25cm] at (0.25,4){$y$};
%
% \node[text width=2cm, align=center] (1) at (2,5){$\omega t= 0$};
% \node[text width=2cm, align=center] (1) at (5,5){$\omega t= \frac{\pi}{2}$};
% \node[text width=2cm, align=center] (1) at (8,5){$\omega t= \pi$};
% \node[text width=2cm, align=center] (1) at (11,5){$\omega t= \frac{3\pi}{2}$};
% \node[text width=2cm, align=center] (1) at (14,5){$\omega t= 2\pi$};
% \begin{scope}[]
% \draw[dots along my path]  (2,3) ellipse (1 and 1);
% \end{scope}
% \begin{scope}[shift={(3,-0.5)},transform canvas={yscale=1.2}]
% \draw[dots along my path]  (2,3) ellipse (1 and 1);
% \end{scope}
% \begin{scope}[shift={(6,0)}]
% \draw[dots along my path]  (2,3) ellipse (1 and 1);
% \end{scope}
% \begin{scope}[shift={(7.2,0)},transform canvas={xscale=1.2},rotate around
% ={45,(9.2,3)}] \draw[dots along my path]  (2,3) ellipse (1 and 1);
% \end{scope}
% \begin{scope}[shift={(12,0)}]
% \draw[dots along my path]  (2,3) ellipse (1 and 1);
% \end{scope}
% \end{tikzpicture}
% \caption{Gravitational wave.}
% \end{figure}


\section{Degrees of Freedom (DoF) and Scalar Vector Tensor (SVT) Decomposition in Space/Time}

Parametrize the line element as
\begin{equation}
    \dif s^2 = - \left(1+2\Phi\right)\dif t^2 + \tensor{v}{_i} \left(\dif t \dif \tensor{x}{^i} + \dif t \dif \tensor{x}{^i}\right) + \left(\tensor{\delta}{_{ij}} + \tensor{h}{_{ij}} \right)\dif \tensor{x}{^i}\dif \tensor{x}{^j}\,.
\end{equation}
Block matrix
\begin{equation}
    \tensor{h}{_{\mu\nu}} =
    \begin{bmatrix}
        \tensor{h}{_{00}} & \tensor{h}{_{0j}} \\
        \tensor{h}{_{i0}} & \tensor{h}{_{ij}}
    \end{bmatrix}
    =
    \begin{bmatrix}
        -2\Phi & \tensor{v}{_{j}} \\
        \tensor{v}{_{i}} & \tensor{h}{_{ij}}
    \end{bmatrix}
    \, , \quad \abs{\Phi},\abs{\tensor{v}{_i}},\abs{\tensor{h}{_{ij}}} \ll 1\,.
\end{equation}
In general algebraic decomposition: symmetric/antisymmetric
\begin{equation}
    \tensor{T}{_{\mu\nu}} = \tensor*{T}{_{\mu\nu}^{\text{s}}} + \tensor*{T}{_{\mu\nu}^{\text{as}}} = \tensor{T}{_{(\mu\nu)}} + \tensor{T}{_{[\mu\nu]}} = \frac{1}{2}\left(\tensor{T}{_{\mu\nu}}+\tensor{T}{_{\nu\mu}}\right) + \frac{1}{2}\left(\tensor{T}{_{\mu\nu}}-\tensor{T}{_{\nu\mu}}\right)\,.
\end{equation}
Can we decompose $\tensor*{T}{_{\mu\nu}^{\text{s}}}$ further?\newline
Wake the trace
\begin{equation}
    \tensor{T}{^{\text{s}}} = \tensor{g}{^{\mu\nu}}\tensor*{T}{_{\mu\nu}^{\text{s}}}\, ,
\end{equation}
so we can write $\tensor*{T}{_{\mu\nu}^{\text{s}}}$ as follows:
\begin{equation}
    \tensor*{T}{_{\mu\nu}^{\text{s}}} = \tensor*{T}{_{\mu\nu}^{\text{tf}}} + \frac{1}{d}\,\tensor{g}{_{\mu\nu}}\tensor{T}{^{\text{s}}}\,,
\end{equation}
where $d$ denotes the dimension of spacetime and $\tensor*{T}{_{\mu\nu}^{\text{tf}}}$ is a tracefree, symmetric tensor.
Can we decompose $\tensor*{T}{_{\mu\nu}^{\text{tf}}}$ any further? Possible for tensor fields $\tensor{T}{_{\mu\nu}}(x)$ as this involves derivatives.
Decompose metric perturbations:
\begin{equation}
    \tensor{h}{_{00}} = -2\Phi
\end{equation}
\begin{equation}
    \tensor{h}{_{0i}} = \tensor{v}{_i}
\end{equation}
\begin{equation}
    \tensor{h}{_{ij}} = 2\tensor{s}{_{ij}} - 2\Psi\tensor{s}{_{ij}}\,\quad
    \begin{cases}
\Psi := -\frac{1}{6}\tensor{\delta}{^{ij}}\tensor{h}{_{ij}}\,\quad
&\text{``trace''} \\
\tensor{s}{_{ij}} := \frac{1}{2} \left(\tensor{h}{_{ij}} - \frac{1}{3}\tensor{\delta}{^{kl}}\tensor{h}{_{kl}}\tensor{\delta}{_{ij}}\right)\,\quad &\text{``strain''}
\end{cases}
\end{equation}
\begin{equation}
    \dif s^2 = -\left(1+2\Phi\right)\dif t^2 + \tensor{v}{_i} \left(\dif t \dif \tensor{x}{^i} + \dif t \dif \tensor{x}{^i}\right) + \left[\left(1-2\Psi\right)\tensor{\delta}{_{ij}} + 2 \tensor{S}{_{ij}} \right]\dif \tensor{x}{^i}\dif \tensor{x}{^j}
\end{equation}
$\tensor{\partial}{_i} = \tensor{k}{_i}$ (momentum space)
Decompose vector $\tensor{\omega}{_i}$ into transverse and longitudinal components.
\begin{align}
    \tensor{v}{_i} = &\tensor{S}{_i} + \tensor{\partial}{_i}B\\
    &3\,\,+\,\,\,\,\,1\,\,\,\,\underbrace{-1}_{\tensor{\partial}{_i}\tensor{S}{^i}=0}\,\quad \text{DoF}
\end{align}
\begin{equation}
    \tensor{S}{_{ij}} = \tensor{\partial}{_{(i}}\tensor{F}{_{j)}} + \left( \tensor{\partial}{_i}\tensor{\partial}{_j} - \frac{1}{3}\,\tensor{\delta}{_{ij}}\Delta\right) E + \tensor*{h}{_{ij}^{\text{\tiny TT}}}
\end{equation}
\begin{center}
    \begin{tabular}{c l}
        $4$ & 4 scalars \\
        $+$ & \\
        $2\,(3-1)$ & 2 transverse vectors \\
        $+$ & \\
        $6-1-3$ & 1 symmetric transverse traceless tensor \\
        \midrule
        $10$ & independent components of $\tensor{h}{_{\mu\nu}}$
    \end{tabular}
\end{center}
valid for (reduce free components):
\begin{equation}
    \tensor{\partial}{_i}\tensor{F}{^i}=0\,,\quad\tensor*{h}{_{ij}^{\text{\tiny TT}}}=\tensor*{h}{_{ji}^{\text{\tiny TT}}}\,,\quad\tensor*{h}{_{ij}^{\text{\tiny TT}}}\tensor{\delta}{^{ij}}=0\,,\quad\tensor{\partial}{^i}\tensor*{h}{_{ij}^{\text{\tiny TT}}}=0
\end{equation}

\begin{figure}[hbtp!]
\centering
 \includegraphics{antisymmetricMatrix.pdf}
\caption{}
%TODO Caption
\end{figure}
\begin{figure}[hbtp!]
\centering
 \includegraphics{Coorddist.pdf}
\caption{}
%TODO Caption
\end{figure}

\begin{figure}[hbtp!]
\centering
 \includegraphics[scale=.75]{ellipse1.pdf}
 \includegraphics[scale=.75]{ellipse2.pdf}
\caption{}
%TODO Caption
\end{figure}


\subsection{Decomposition for tensor \emph{fields}}
Tensor fields depend on the space-time coordinate $x$.
We can build tensors from derivatives of one-rank tensors.
%irgendwas mit $\nabla_i = \partial_i$
Each derivative $\partial_i$ corresponds to $k_i$ in momentum space.
$k_i$ points in some direction.
We can decompose e.g.\ $v_i$ with respect to the direction $k_i$.
\begin{equation}
    v_i = v_{i,\perp} + v_{i,\parallel},
\end{equation}
where $v_{i,\perp}$ is perpendicular and $v_{i,\parallel}$ parallel to $k_i$.
The transverse vector is divergence free:
\begin{equation}
    \tensor{\partial}{_i} \tensor{v}{_\perp^i} = 0 \qquad \left( \implies \vec{k} \perp \vec{v} \right)
\end{equation}
The longitudinal vector $\vec{v}_\parallel$ is curl free.
\begin{equation}
    \tensor{\epsilon}{^i^j^k} \tensor{\partial}{_j} \tensor{v}{_\perp_k} = 0
\end{equation}
$\implies$ conditions for differential equations, makes only sense when applied to a vector field.
The transverse vector can be represented as curl of some other vector:
\begin{equation}
    \tensor{v}{^i_\perp} = \tensor{\epsilon}{^i^j^k} \tensor{\partial}{_i} \tensor{u}{_k} \,.
\end{equation}
The choice of $u_i$ is not ??? unless we impose additional conditions on $u$:
\begin{equation}
    \tensor{\partial}{_i} \tensor{u}{^i} = 0 \qquad \left( \text{``divergence free''} \right)
\end{equation}
The longitudinal vector is the divergence of a scalar function $\lambda(x)$
\begin{equation}
    \tensor{v}{_\partial_j} = \tensor{\partial}{_j} \lambda(x)
\end{equation}
Similarly, we can decompose the strain vector $\tensor{s}{_i_j}$:
\begin{equation}
    \tensor{s}{^i^j} = \tensor{s}{_{\perp}^i^j} + \tensor{s}{_s^i^j} + \tensor{s}{_{\parallel}^i^j}
    \qquad \left( \text{symmetric, trace free}, \tensor{s}{_i_j} = \tensor{s}{_j_i}, \tensor{\delta}{_i_j} \tensor{s}{^i^j} \right)
\end{equation}
with the transverse part $\tensor{s}{_\perp^i^j}$, the solenoidal part $\tensor{s}{_s^i^j}$,
and the parallel part $\tensor{s}{_\parallel^i^j}$.
\begin{itemize}
    \item The transverse part is divergence free $ \tensor{\delta}{_i} \tensor{s}{_\perp^i^j} = 0$.
    \item The divergence of the solenoidal part is a transverse vector,
    \begin{equation}
        \tensor{\partial}{_i} \tensor{\partial}{_j} \tensor{s}{_s^i^j} = 0
    \end{equation}
    \item The divergence of the longitudinal part is a longitudinal vector
    \begin{equation}
        \tensor{\epsilon}{^j^k^l} \tensor{\partial}{_l} \tensor{\partial}{_i} \tensor{s}{_\parallel^i_j} = 0
    \end{equation}
\end{itemize}
The longitudinal part can be constructed from a scalar field $E$:
\begin{equation}
    \tensor{s}{^\parallel_i_j} = \left( \tensor{\partial}{_i} \tensor{\partial}{_j} - \frac{1}{3} \tensor{\partial}{_i_j} \Delta \right) E
\end{equation}
The part in the braces is the only object with two derivatives of a scalar with two free indices. 
The coefficients are fixed by $ \tensor{\delta}{^i^j} \tensor{s}{^\parallel_i_j} = 0$ (trace free).
The solenoidal part can be constructed from a traversal vector field $F_i$.

\chapter{The Schwarzschild Solution}
\begin{figure}[hbtp!]
\centering
\includegraphics{gravlens.pdf}
\caption{}
%TODO Caption
%TODO Position
\end{figure}

\begin{figure}[hbtp!]
\centering
 \includegraphics{lightdeflection.pdf}
\caption{}
%TODO Caption
%TODO position
\end{figure}


\chapter{Experimental Tests in the Solar System}
How can we test relativistic effects in our solar system?
\begin{itemize}
  \item The sum of the mass of all planets is much smaller than the solar mass
  $M_{\astrosun}\approx \unit[2\cdot 10^{30}]{kg}$. The heaviest planet is
  Jupiter with a mass of $M_{\jupiter}\approx \unit[2\cdot 10^{27}]{kg}$,
  therefore we can assume the planets to be testparticles.
  \item The sun is in good approximation a spherically symmetric object. We can
  therefore use the Schwarzschild metric.
\end{itemize}
Consider the variation of the energy functional 
\begin{equation}
\delta
\int\tensor{g}{_\mu_\nu}\tensor{\dot{x}}{^\mu}\tensor{\dot{x}}{^\nu}\dif\lambda
\end{equation}
We define
$K:=-\tensor{g}{_\mu_\nu}\tensor{\dot{x}}{^\mu}\tensor{\dot{x}}{^\nu}$, which is
conserved along geodesics and it holds true that
\begin{equation}
K=\begin{cases}
-1& \mathrm{\ timelike\ geodesics}\\
\phantom{-}0& \mathrm{\ lightlike\ geodesics}
\end{cases}
\end{equation}
We can explicitly write
\begin{equation}
K=-e^{2a(r)}\dot{t}^2+e^{2b(r)}\dot{r}^2+r^2\left(\dot{\vartheta}+\sin^2\vartheta
\dot{\phi}\right)
\end{equation}
A Killing-vector $\tensor{\xi}{^\mu}$ satisfies
\begin{equation}
\tensor{\xi}{_\mu}\tensor{\dot{x}}{^\mu}=\mathrm{\ const.}
\end{equation}
For the Schwarzschild metric, there are four independent Killing-vectors,
corresponding to 3 rotations, and staticity (time independence).
Conservation of angular momentum leads to a motion in a plane, w.l.o.g.\ we can
chose a coordinate system in which $\vartheta=\nicefrac{\pi}{2}$.
The Killing-Vectors are given by
\begin{align}
\tensor*{\xi}{_{(\varphi)}^\mu}&=(\partial_\varphi)^\mu=\tensor*{\delta}{_\varphi^\mu}\,,\\
\tensor*{\xi}{_{(t)}^\mu}&=(\partial_t)^\mu=\tensor*{\delta}{_t^\mu}\,.\\
\end{align}
The associated conserved quantities are
\begin{align}
E&:=\tensor*{\xi}{_{(\varphi)}^\mu}\tensor{g}{_\mu_\nu}\tensor{\dot{x}}{^\nu}
=\tensor{g}{_t_t}\dot{t}
=e^{2a}\dot{t}
=\left(1-\frac{2M}{r}\right)\dot{t}
\\
L&:=\tensor*{\xi}{_{(t)}^\mu}\tensor{g}{_\mu_\nu}\tensor{\dot{x}}{^\nu}
=\tensor{g}{_\varphi_\varphi}\dot{\varphi}
=r^2\dot{\varphi}\,.
\end{align}
For massless
%TODO stimmt das?
particles we can think of $E$ and $L$ as conserved energy and angular momentum.
\begin{equation}
K=-\left(1-\frac{2M}{r}\right)\dot{t}^2+\left(1-\frac{2M}{r}\right)^{-1}\dot{r}^2+r^2\varphi^2
\end{equation}
If we insert the conserved quantities $L,E$ and multiply with
$\left(1-\frac{2M}{r}\right)$, we get
\begin{equation}
\frac{E^2}{2}=\frac{\dot{r}^2}{2}
+\left(1-\frac{2M}{r}\right)\left(\frac{L^2}{2r}-\frac{K}{2}\right)\,.
\end{equation}
This expression can be rearanged to the Form
\begin{equation}
\frac{\dot{r}^2}{2}+V\textsubscript{eff}(r)
=\varepsilon\,,
\end{equation}
where we defined
\begin{equation}
V\textsubscript{eff}(r):=\frac{MK}{r}+\frac{L^2}{2r^2}-\frac{ML^2}{r^3}\,,\quad
\varepsilon:=\frac{E^2+K}{2}\,.
\end{equation}
\section{Perihelion Shift of Mercury}
It has been known for a long time that the perihelion of the mercury moves by
roughly $5061''\footnote{Where an arcsec $''$ denotes the 3600 part of
an degree.}/\textrm{century}$, if on subtracts other effects such a
fraction of $43''/\textrm{century}$. 
There where several proposed explanations for this including
\begin{itemize}
  \item a new planet called vulcan between mercury and the sun
  \item a change to newtons $1/r^2$-law 
  \item effects due the suns quadrupole moment 
\end{itemize}
We will now look at what is predicted by GR.

\chapter{Black Holes}
\begin{figure}[hbtp!]
\centering
 \includegraphics{fusion.pdf}
\caption{}
%TODO Caption
\end{figure}
\begin{figure}[hbtp!]
\centering
 \includegraphics{Nucleids.pdf}
\caption{}
%TODO Caption
\end{figure}
\begin{figure}[hbtp!]
\centering
 \includegraphics{nuclpot.pdf}
\caption{}
%TODO Caption
\end{figure}
\begin{figure}[hbtp!]
\centering
 \includegraphics{plot4.pdf}
\caption{}
%TODO Caption
\end{figure}
\begin{figure}[hbtp!]
\centering
 \includegraphics{pressurevsgravity.pdf}
\caption{}
%TODO Caption
\end{figure}

\chapter{Cosmology}
\begin{figure}[hbtp!]
\centering
 \includegraphics{spatialstruct3.pdf}\hspace{-1.5cm}
 \includegraphics{spatialstruct2.pdf}\hspace{-1.5cm}
 \includegraphics{spatialstruct1.pdf}
\caption{Sectional curvature $\kappa$ and shapes of the spatial part
of spacetime.}
\end{figure}


\end{document}
