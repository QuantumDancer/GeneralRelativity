\chapter{Gravitational Waves}
\begin{itemize}
    \item vacuum solution of linearized Einstein equations
    \begin{equation}
        \Box \overline{h}_{\mu\nu} = 0 \qquad \text{in de Donder gauge}
    \end{equation}
    \item describes \emph{weak} gravitational waves \emph{only} $\rightarrow$ linearized treatment justified
    \item description breaks down for strong gravitational fields as the theory becomes essentially \emph{non-linear} (e.g.\ two black holes merge)
    \item analysis similar to electrodynamics, but here $h_{\mu\nu}$: spin-2 field, $A_\mu$: spin-1 field
    \item vacuum equations
    \begin{equation}
        \Box \overline{h}_{\mu\nu} = 0 \qquad \overline{h}_{\mu\nu}=h_{\mu\nu} - \frac{1}{2} \eta_{\mu\nu} h \qquad \partial^\mu \overline{h}_{\mu\nu} = 0
    \end{equation}
    \item Gauge freedom not yet completely exhausted by de Donder gauge.
    \begin{equation}
        \partial^\mu \overline{h'}_{\mu\nu} = \underbrace{\partial^\mu \overline{h'}_{\mu\nu}}_{=0} + \Box \xi_\nu = 0
    \end{equation}
    $\implies$ All gauge transformations generated by $\xi_\nu$ that satisfy $\Box \xi_\nu = 0$ do not lead out of the de Donder gauge. \\
    Compare with electromagnetic field:
    \begin{align}
        & A_\mu \to {A'}^\mu = A^\mu + \partial_\mu \lambda(x) \\
        & \partial_\mu A^\mu = 0 \qquad \text{Lorentz gauge} \\
        & \partial_\mu {A'}^\mu = \underbrace{ \partial_\mu A^\mu}_{=0} + \partial_\mu \partial^\mu \lambda(x) = 0 \implies \Box \lambda = 0
    \end{align}
    Exploit this remaining gauge freedom to make perturbations $h_{\mu\nu}$ \emph{transverse} and \emph{traceless}
    \item transversality
    \begin{equation}
        \partial^\mu h'_{\mu\nu} = \partial^\mu h_{\mu\nu} + \Box \xi_\nu + \partial_\nu \partial^\mu \xi_\mu \overset{!}{=} 0
    \end{equation}
    Since only gauge transformations that satisfy $\Box \xi_\nu$ are allowed (do not lead out of de Donder gauge), the equation that shoud
    be solved for $x_\nu$ is
    \begin{equation}
        \partial_\nu \partial^\mu \xi_\mu = - \partial^\mu h_{\mu\nu}
    \end{equation}
    \item In order for the perturbations to be traceless, we need to find a solution to
    \begin{equation}
        h' = h + \partial_\mu \xi^\mu = 0 \implies \partial_\mu x^\mu = -\frac{1}{2}h
    \end{equation}
\end{itemize}
\begin{figure}
\centering
\begin{tikzpicture}[decoration={markings,
  mark=between positions 0 and 1 step 22.3pt
  with { \draw [fill] (0,0) circle [radius=2pt];}}]
\draw[->] (0,1) -- (3,1);
\draw[->] (0,1) -- (0,4); 
\node[text width=0.25cm] at (3,1.25){$x$};
\node[text width=0.25cm] at (0.25,4){$y$};

\node[text width=2cm, align=center] (1) at (2,5){$\omega t= 0$};
\node[text width=2cm, align=center] (1) at (5,5){$\omega t= \frac{\pi}{2}$};
\node[text width=2cm, align=center] (1) at (8,5){$\omega t= \pi$};
\node[text width=2cm, align=center] (1) at (11,5){$\omega t= \frac{3\pi}{2}$};
\node[text width=2cm, align=center] (1) at (14,5){$\omega t= 2\pi$};
\begin{scope}[]
\draw[dots along my path]  (2,3) ellipse (1 and 1);
\end{scope}
\begin{scope}[shift={(3,-0.5)},transform canvas={yscale=1.2}]
\draw[dots along my path]  (2,3) ellipse (1 and 1);
\end{scope}
\begin{scope}[shift={(6,0)}]
\draw[dots along my path]  (2,3) ellipse (1 and 1);
\end{scope}
\begin{scope}[shift={(7.2,0)},transform canvas={xscale=1.2}]
\draw[dots along my path]  (2,3) ellipse (1 and 1);
\end{scope}
\begin{scope}[shift={(12,0)}]
\draw[dots along my path]  (2,3) ellipse (1 and 1);
\end{scope}
\end{tikzpicture}
\caption{Gravitational wave.}
\end{figure}