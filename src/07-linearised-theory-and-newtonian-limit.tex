\chapter{Linearized theory and Newtonian limit}
\section{Linearized theory}
Consider a weak gravitational field. Then we can split the full spacetime metric $g_{\mu\nu}(x)$ into two parts.
\begin{definition}{Linearization of the metric field.}
\begin{equation}
    g_{\mu\nu}(x) = \eta_{\mu\nu} + h_{\mu\nu}(x) + \landauO(h^2) \, .
\end{equation}
\end{definition}
$\eta_{\mu\nu}$ is the flat, constant ``background'' metric of Minkowski space, i.e.\ there is no gravitational field present.
The field $h_{\mu\nu}(x)$ can be interpreted as a perturbation on the fixed background $\eta_{\mu\nu}$.
One can identify a spin-2 particle, the so-called \emph{graviton}, with the excitations (quantized fluctuations) of this field.
Because only the linear order of $h$ is considered, the nonlinearity of Einstein's equations is lost.
We can raise and lower indices with $\eta_{\mu\nu}$ and $\eta^{\mu\nu}$.

\begin{remark}
This works only for a weak gravitational field, since a strong gravitational field produces a strong back reaction of ``matter''
on the geometry, which follows from the nonlinearity of Einstein's equations.
Exactly this back reaction is neglected in the linearized theory.
\end{remark}

\subsection{Derivation of the linearized Einstein's equations}
In the following we neglect all terms with $\landauO(h^2)$.
Our goal is to express Einstein's field equations in the linearized approximation.
For this we need to calculate the Christoffel symbols, the Riemann tensor, the Ricci tensor and the Ricci scalar.

Christoffel symbols of the first kind:
\begin{equation}
    \csym{\mu}{\nu}{\rho} = \frac{1}{2} \left( \tensor{h}{_\mu_\rho_,_\nu} + \tensor{h}{_\nu_\rho_,_\mu}
    - \tensor{h}{_\mu_\nu_,_\rho} \right) + \landauO(h^2) \, .
\end{equation}
Christoffel symbols of the second kind:
\begin{equation}
    \cSym{\rho}{\mu}{\nu} = g^{\rho\sigma} \csym{\mu}{\nu}{\sigma} = \eta^{\rho\sigma} \csym{\mu}{\nu}{\sigma} + \landauO(h^2)
    = \frac{1}{2} \left( \tensor{h}{_\mu^\rho_,_\nu} + \tensor{h}{_\nu^\rho_,_\mu} - \tensor{h}{_\mu_\nu^,^\rho} \right) + \landauO(h^2) \, .
\end{equation}
Riemann tensor:
\begin{equation}
    \begin{split}
        \tensor{R}{^\rho_\sigma_\mu_\nu}
        &= \partial_\mu \cSym{\rho}{\nu}{\sigma} - \partial_\nu \cSym{\rho}{\mu}{\sigma}
        + \underbrace{\cSym{\rho}{\mu}{\lambda} \cSym{\lambda}{\nu}{\sigma} - \cSym{\rho}{\nu}{\lambda} \cSym{\lambda}{\mu}{\sigma}}_{\landauO(h^2)} \\
        &= \frac{1}{2} \left( \tensor{h}{_\nu^\rho_,_\sigma_\mu} + \mathunderline{blue}{\tensor{h}{^\rho_\sigma_,_\nu_\mu}} - \tensor{h}{_\nu_\sigma^,^\rho_\mu}
        - \mathunderline{blue}{\tensor{h}{^\rho_\mu_,_\sigma_\nu}} + \tensor{h}{^\rho_\sigma_,_\mu_\nu} + \tensor{h}{_\mu_\sigma^,^\rho_\nu} \right) + \landauO(h^2) \\
        &= \frac{1}{2} \left( \tensor{h}{_\nu^\rho_,_\sigma_\mu} - \tensor{h}{_\nu_\sigma^,^\rho_\mu}
        - \tensor{h}{^\rho_\sigma_,_\mu_\nu} + \tensor{h}{_\mu_\sigma^,^\rho_\nu} \right) + \landauO(h^2) \, .
    \end{split}
\end{equation}
Ricci tensor
\begin{equation}
    \tensor{R}{_\sigma_\nu} = \tensor{R}{^\rho_\sigma_\rho_\nu} 
    = \frac{1}{2} \left( \tensor{h}{_\nu^\rho_,_\sigma_\rho} - \tensor{h}{_\nu_\sigma^,^\rho_\rho} 
    - \tensor{h}{_,_\sigma_\nu} + \tensor{h}{_\rho_\sigma^,^\rho_\nu} \right) + \landauO(h^2)\, ,
\end{equation}
with $h\coloneqq h_{\mu\nu}\eta^{\mu\nu}$

Ricci scalar
\begin{equation}
    R = g^{\sigma\nu}R_{\sigma\nu} = \eta^{\sigma\nu}R_{\sigma\nu} + \landauO(h^2) 
    = \tensor{h}{^\sigma^\nu_,_\sigma_\nu} - \tensor{h}{_,_\sigma^\sigma} + \landauO(h^2) 
\end{equation}
We define 
\begin{equation}
    \overline{h}_{\mu\nu} \coloneqq h_{\mu\nu} - \frac{1}{2} \eta_{\mu_\nu}h\,.
\end{equation}
With the following two lines we can show that $\overline{\overline{h}} = h$:
\begin{align}
    \overline{h} \coloneqq &\ \overline{h}_{\mu\nu}\eta^{\mu\nu} = h - 2h = -h \\
    h_{\mu_\nu} =&\ \overline{h}_{\mu\nu} + \frac{1}{2} \eta_{\mu\nu}h = \overline{h}_{\mu\nu} - \frac{1}{2} \eta_{\mu\nu}\overline{h}
\end{align}

\subsubsection{Linearized Einstein tensor \texorpdfstring{$G_{\mu\nu}$}{Gmunu} in terms of \texorpdfstring{$\overline{h}_{\mu\nu}$}{hbarmunu}}

\begin{equation}
    \begin{split}
        G_{\mu\nu}^{\text{(L)}} =\ & R_{\mu\nu}^{\text{(L)}} - \frac{1}{2} \eta_{\mu\nu} R^{\text{(L)}} \\
        =\ & \frac{1}{2} \partial_\mu \partial_\rho \tensor{h}{_\nu^\rho} + \frac{1}{2} \partial_\nu \partial_\rho \tensor{h}{_\mu^\rho}
        - \frac{1}{2} \Box h_{\mu\nu} - \frac{1}{2} \partial_{\mu\nu}h-\frac{1}{2} \eta_{\mu\nu}\partial_\rho\partial_\sigma h^{\rho\sigma}
        + \frac{1}{2} \eta_{\mu_\nu}\Box h \\
        =\ & \frac{1}{2} \partial_\mu\partial_\rho \tensor{\overline{h}}{_\nu^\rho} 
        - \mathunderline{blue}{\frac{1}{4}\partial_\mu\partial_\nu\overline{h}} 
        + \frac{1}{2} \partial_\nu\partial_\rho\tensor{\overline{h}}{_\mu^\rho} 
        - \mathunderline{blue}{\frac{1}{4}\partial_\nu\partial_\mu\overline{h}} - \frac{1}{2}\Box\overline{h}_{\mu\nu} \\
        & + \mathunderline{green}{\frac{1}{2}\eta_{\mu\nu}\Box\overline{h}} + \mathunderline{blue}{\frac{1}{2}\partial_\mu\partial_\nu\overline{h}}
        - \frac{1}{2}\eta_{\mu\nu}\partial_\rho\partial_\sigma\overline{h}^{\rho\sigma}
        + \mathunderline{green}{\frac{1}{4}\eta_{\mu\nu}\Box\overline{h}} - \mathunderline{green}{\frac{1}{2}\eta_{\mu\nu}\Box\overline{h}} \\
        =\ & -\frac{1}{2} \Box \overline{h}_{\mu\nu} + \partial_\rho \tensor{\partial}{_(_\mu}\tensor{\overline{h}}{_\nu_)^\rho} 
        - \frac{1}{2} \eta_{\mu\nu}\partial_\rho\partial_\sigma \overline{h}^{\rho\sigma} \\
        \overset{!}{=}\ & \kappa T_{\mu\nu}
    \end{split}
\end{equation}
with the (linearized )d'Alembert operator 
\begin{equation}
    \Box^{\text{(L)}} = \Box = \partial_\mu\partial_\nu \eta^{\mu\nu}=\partial_\mu\partial^\mu
\end{equation}

\subsubsection{Gauge transformations}
Usually field equations are in the form of
\begin{equation}
    \Box \text{``field''} = \text{``source''}
\end{equation}